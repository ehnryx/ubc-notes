\documentclass[11pt]{article}
\usepackage{hyperref}
\usepackage[margin=1in]{geometry}
\usepackage{amsmath}
\usepackage{amsthm}
\usepackage{amssymb}
\usepackage{amsfonts}
\usepackage{graphicx} %\usepackage[pdftex]{graphicx}
%\usepackage{tikz}


\setlength\parindent{0pt}

\newtheorem{thm}{Theorem}[section]
\newtheorem{cor}{Corollary}[thm]
\newtheorem{lemma}[thm]{Lemma}
\theoremstyle{definition}
\newtheorem{defn}{Definition}[section]
\newtheorem{example}{Example}[section]
\newtheorem{prop}{Proposition}[section]
\newtheorem{pty}{Property}[section]
\newtheorem{remark}{Remark}[section]
\newtheorem{obs}{Observation}[section]
\newcommand{\The}[2]{\begin{#1}#2\end{#1}}

\newcommand{\ord}[0]{\text{ord}}

% notes
\iftrue 
\newcommand{\f}[2]{\frac{#1}{#2}}
\newcommand{\re}[1]{\frac{1}{#1}}
\newcommand{\half}[0]{\frac{1}{2}}
\newcommand{\ift}[0]{It follows that}
\newcommand{\cp}[1]{\overline{#1}}
\newcommand{\Note}[0]{\noindent\textbf{Note: }} 
\newcommand{\Claim}[0]{\noindent\textbf{Claim: }} 
\newcommand{\Lemma}[1]{\noindent\textbf{Lemma #1}: } %
\newcommand{\Ex}[0]{\noindent\textbf{Example: }} %
\newcommand{\Special}[0]{\noindent\textbf{Special case: }} %
\newcommand{\solution}[2]{\item[]\proof[Solution to #1] #2 \qedhere}
\newcommand{\legendre}[2]{\left(\frac{#1}{#2}\right)}
\newcommand{\dent}[0]{\hspace{0.5in}}
\fi

\newcommand{\sm}[0]{\setminus}
\newcommand{\set}[1]{\left\{ #1 \right\}}
\newcommand{\nl}[0]{\vspace{12pt}}
\newcommand{\rng}[2]{#1,\dots,#2}
\newcommand{\srng}[3]{#1_#2,\dots,#1_#3}
\newcommand{\st}[0]{\text{ such that }}
\newcommand{\et}[0]{\text{ and }}
\newcommand{\then}[0]{\text{ then }}
\newcommand{\forsome}[0]{\text{ for some }}
\newcommand{\floor}[1]{\lfloor #1 \rfloor}

% misc
\newcommand{\abs}[1]{\left\lvert#1\right\rvert} %
% lcm ???
\DeclareMathOperator{\lcm}{lcm} 
% blackboard bold
\newcommand{\RR}{\mathbb{R}}
\newcommand{\FF}{\mathbb{R}}
\newcommand{\QQ}{\mathbb{Q}}
\newcommand{\ZZ}{\mathbb{Z}}
\newcommand{\NN}{\mathbb{N}}
\newcommand{\CC}{\mathbb{C}}
% mathcal
\newcommand{\m}[1]{\mathcal{#1}}
% vectors
\newcommand{\vvec}[1]{\textbf{#1}} %
\newcommand{\ii}[0]{\vvec{i}} %
\newcommand{\jj}[0]{\vvec{j}} %
\newcommand{\kk}[0]{\vvec{k}} %
\newcommand{\hvec}[1]{\hat{\textbf{#1}}} %
\newcommand{\cvec}[3]{ %column vector
    \ensuremath{\left(\begin{array}{c}#1\\#2\\#3\end{array}\right)}}
\newcommand{\pfrac}[2]{\frac{\partial#1}{\partial#2}} %
\newcommand{\norm}[1]{\left\lVert#1\right\rVert} %
% dotp roduct
\makeatletter
\newcommand*\dotp{\mathpalette\dotp@{.5}}
\newcommand*\dotp@[2]{\mathbin{\vcenter{\hbox{\scalebox{#2}{$\m@th#1\bullet$}}}}}
\makeatother
% divrg and curl
\newcommand{\divrg}[0]{\nabla\dotp} %
\newcommand{\curl}[0]{\nabla\times} %

\title{Math 322 Notes}
\author{Henry Xia}
%\date{15 September 2017}

\begin{document}

\maketitle

\tableofcontents

%%%%%%%%%%%%%%%%%%%%%%%%%%%%%%%%%%%%%%%%%%%%%%%%%%%%%%%%%%%%%%%%%%%%%%
%%%%%%%%%%%%%%%%%%%%%%%%%%%%%%%%%%%%%%%%%%%%%%%%%%%%%%%%%%%%%%%%%%%%%%
% 2018 09 06

\section{Introduction}

\begin{defn}
	A \emph{quotient set} is a set $S/\sim$ whose elements are in one to one correspondence
	with equivalence classes of $S$. We also write $\overline{S} = S/\sim$.
\end{defn}

\begin{defn}
	A \emph{natural map} $S\to S/\sim$ is a surjective map to the equivalence classes of
	$S$. 
\end{defn}


%%%%%%%%%%%%%%%%%%%%%%%%%%%%%%%%%%%%%%%%%%%%%%%%%%%%%%%%%%%%%%%%%%%%%%
%%%%%%%%%%%%%%%%%%%%%%%%%%%%%%%%%%%%%%%%%%%%%%%%%%%%%%%%%%%%%%%%%%%%%%
% 2018 09 11

\begin{defn}
	A \emph{semigroup} is a set $S$ that is closed under multiplication. That is $\forall
	a,b\in S$, we have $ab\in S$. 
\end{defn}

\begin{defn}
	A \emph{monoid} is a semigroup that has an identity element. That is $\exists 1\in S$ such
	that $1a = a1 = a$ for all $a\in S$. 
\end{defn}

\begin{defn}
	A \emph{group} is a monoid where every element has an inverse. That is $\forall a\in S$,
	there exists some $a^{-1}$ such that $aa^{-1} = 1$. 
\end{defn}

\begin{defn}
	A \emph{subgroup} is a subset of a group that is also a group.
\end{defn}

\begin{defn}
	An \emph{Abelian group} is a group whose multiplication is commutative. 
\end{defn}

\begin{defn}
	The \emph{symmetric group} on $n$ elements is the set of all permutations of $n$
	elements. We denote this as $S_n$. 
\end{defn}

\begin{defn}
	A \emph{cyclic group} is a group that can be generated by one of its elements. That is
	$G = \set{a, a^2, \dots, a^{n-1}, a^n=1}$. We say that $a$ generates $G$.
\end{defn}

%%%%%%%%%%%%%%%%%%%%%%%%%%%%%%%%%%%%%%%%%%%%%%%%%%%%%%%%%%%%%%%%%%%%%%
%%%%%%%%%%%%%%%%%%%%%%%%%%%%%%%%%%%%%%%%%%%%%%%%%%%%%%%%%%%%%%%%%%%%%%
% 2018 09 13

\begin{defn}
	The \emph{order} of a group is the number of elements in the cardinality of the group.
	The order of an element $a$ is the smallest $m$ such that $a^m = 1$. If no such $m$
	exists, we say $a$ has infinite order. This is equivalently the order of the group
	generated by $a$. 
\end{defn}

\begin{defn}
	The \em{direct product} is the cartesian product of the groups, where the
	group action is defined componentwise.
\end{defn}

\begin{defn}
	We say a group $G$ is isomorphic to a group $H$, or $G\simeq H$, if there exists a
	bijection from $G$ to $H$ that preserves the group action. That is there exists some
	bijection $f:G\to H$ such that $f(xy) = f(x)f(y)$.
\end{defn}

\begin{example}
	The group $\RR$ with addition is isomorphic to $\RR^*_+$ with multiplication. We can
	take $f(x)=e^x$, then $f(x+y) = e^{x+y} = e^x e^y = f(x)f(y)$. 
\end{example}

\begin{thm} [Cayley's Theorem]
	Any finite group $G$ is isomorphic to a subgroup of the symmetric group acting on $G$.
\end{thm}
\proof
Let $G = \set{x_1,x_2,\dots,x_n}$ have order $n$. Then for each $a\in G$, define
$f_a:G\to G$ where $f_a(x) = ax$. We claim that $f_a$ is a bijection. It suffices to
show that $f_a$ is an injection. Suppose that $f_a(x) = f_a(y)$, then $ax=ay \implies
a^{-1}ax=a^{-1}ay \implies x=y$.

Let $\phi:G\to S_n$ map each element $a\in G$ to the element of $S_n$ that corresponds to
$f_a$. Now we need to check that $\phi$ is injective. Indeed, if $\phi(a) = \phi(b)$, then
$ax = f_a(x) = f_b(x) = bx \implies a=b$. We also need to check that $\phi(ab) =
\phi(a)\phi(b)$. Indeed, $\phi(ab)$ maps $x$ to $abx$, $\phi(b)$ maps $x$ to $bx$, and
$\phi(a)$ maps $bx$ to $abx$.
\qedhere

\begin{example}
	Is $(\QQ,+)$ isomorphic to $(\QQ^*_+,\times)$?

	No. Consider $2x=a$, where $a\in\QQ$. There exists some $x\in\QQ$ for all $a$. This 
	equation becomes $f(x)^2=f(a)$ should the two groups be isomorphic, however, it is
	clear that $f(x)$ does not exist for all $f(a)$. 
\end{example}

\begin{example}
	Fix $a\in G$, then let $C = \set{b\in G : ab=ba}$. We call $a$ the centralizer.
	Observe that $C$ is a subgroup of $G$. 

	It is obvious that $1\in C$.

	Let $x,y\in C$, then $xya = xay = axy$ by associativity, hence $xy\in C$. 

	Let $x\in C$, then $x^{-1}a = x^{-1}axx^{-1} = x^{-1}xax^{-1} = ax^{-1}$ by
	associativity, hence $x^{-1}\in C$. 
\end{example}

\begin{defn}
	The \emph{center} of a group $G$ is the subgroup $\set{a:ax=xa ~~\forall x\in G}$. 
\end{defn}

\begin{remark}
	The intersection of subgroups is also a subgroup. 
\end{remark}

%%%%%%%%%%%%%%%%%%%%%%%%%%%%%%%%%%%%%%%%%%%%%%%%%%%%%%%%%%%%%%%%%%%%%%
%%%%%%%%%%%%%%%%%%%%%%%%%%%%%%%%%%%%%%%%%%%%%%%%%%%%%%%%%%%%%%%%%%%%%%
% 2018 09 18




%%%%%%%%%%%%%%%%%%%%%%%%%%%%%%%%%%%%%%%%%%%%%%%%%%%%%%%%%%%%%%%%%%%%%%
%%%%%%%%%%%%%%%%%%%%%%%%%%%%%%%%%%%%%%%%%%%%%%%%%%%%%%%%%%%%%%%%%%%%%%
% 2018 09 20




%%%%%%%%%%%%%%%%%%%%%%%%%%%%%%%%%%%%%%%%%%%%%%%%%%%%%%%%%%%%%%%%%%%%%%
%%%%%%%%%%%%%%%%%%%%%%%%%%%%%%%%%%%%%%%%%%%%%%%%%%%%%%%%%%%%%%%%%%%%%%
% 2018 09 23




%%%%%%%%%%%%%%%%%%%%%%%%%%%%%%%%%%%%%%%%%%%%%%%%%%%%%%%%%%%%%%%%%%%%%%
%%%%%%%%%%%%%%%%%%%%%%%%%%%%%%%%%%%%%%%%%%%%%%%%%%%%%%%%%%%%%%%%%%%%%%
% 2018 09 25





\end{document}
