\documentclass[11pt]{article}
\usepackage{hyperref}
\usepackage[margin=1in]{geometry}
\usepackage{amsmath}
\usepackage{amsthm}
\usepackage{amssymb}
\usepackage{amsfonts}
\usepackage{graphicx} %\usepackage[pdftex]{graphicx}
\usepackage{xcolor}
%\usepackage{tikz}


\setlength\parindent{0pt}

\newtheorem{thm}{Theorem}[section]
\newtheorem{cor}{Corollary}[thm]
\newtheorem{lemma}[thm]{Lemma}
\theoremstyle{definition}
\newtheorem{defn}{Definition}[section]
\newtheorem{example}{Example}[section]
\newtheorem{prop}{Proposition}[section]
\newtheorem{pty}{Property}[section]
\newtheorem{remark}{Remark}[section]
\newtheorem{obs}{Observation}[section]
\newcommand{\The}[2]{\begin{#1}#2\end{#1}}

\newcommand{\ord}[0]{\text{ord}}

% notes
\iftrue 
\newcommand{\f}[2]{\frac{#1}{#2}}
\newcommand{\re}[1]{\frac{1}{#1}}
\newcommand{\half}[0]{\frac{1}{2}}
\newcommand{\ift}[0]{It follows that}
\newcommand{\cp}[1]{\overline{#1}}
\newcommand{\Note}[0]{\noindent\textbf{Note: }} 
\newcommand{\Claim}[0]{\noindent\textbf{Claim: }} 
\newcommand{\Lemma}[1]{\noindent\textbf{Lemma #1}: } %
\newcommand{\Ex}[0]{\noindent\textbf{Example: }} %
\newcommand{\Special}[0]{\noindent\textbf{Special case: }} %
\newcommand{\solution}[2]{\item[]\proof[Solution to #1] #2 \qedhere}
\newcommand{\legendre}[2]{\left(\frac{#1}{#2}\right)}
\newcommand{\dent}[0]{\hspace{0.5in}}
\fi

\newcommand{\sm}[0]{\setminus}
\newcommand{\set}[1]{\left\{ #1 \right\}}
\newcommand{\expect}[1]{\operatorname{E}\left[\,#1\,\right]}
\newcommand{\nl}[0]{\vspace{12pt}}
\newcommand{\rng}[2]{#1,\dots,#2}
\newcommand{\srng}[3]{#1_#2,\dots,#1_#3}
\newcommand{\st}[0]{\text{ such that }}
\newcommand{\et}[0]{\text{ and }}
\newcommand{\then}[0]{\text{ then }}
\newcommand{\forsome}[0]{\text{ for some }}
\newcommand{\floor}[1]{\lfloor #1 \rfloor}

% misc
\newcommand{\abs}[1]{\left\lvert#1\right\rvert} %
% lcm ???
\DeclareMathOperator{\lcm}{lcm} 
% blackboard bold
\newcommand{\RR}{\mathbb{R}}
\newcommand{\FF}{\mathbb{R}}
\newcommand{\QQ}{\mathbb{Q}}
\newcommand{\ZZ}{\mathbb{Z}}
\newcommand{\NN}{\mathbb{N}}
\newcommand{\CC}{\mathbb{C}}
% mathcal
\newcommand{\m}[1]{\mathcal{#1}}
% vectors
\newcommand{\vvec}[1]{\textbf{#1}} %
\newcommand{\ii}[0]{\vvec{i}} %
\newcommand{\jj}[0]{\vvec{j}} %
\newcommand{\kk}[0]{\vvec{k}} %
\newcommand{\hvec}[1]{\hat{\textbf{#1}}} %
\newcommand{\cvec}[3]{ %column vector
    \ensuremath{\left(\begin{array}{c}#1\\#2\\#3\end{array}\right)}}
\newcommand{\pfrac}[2]{\frac{\partial#1}{\partial#2}} %
\newcommand{\norm}[1]{\left\lVert#1\right\rVert} %
% dotp roduct
\makeatletter
\newcommand*\dotp{\mathpalette\dotp@{.5}}
\newcommand*\dotp@[2]{\mathbin{\vcenter{\hbox{\scalebox{#2}{$\m@th#1\bullet$}}}}}
\makeatother
% divrg and curl
\newcommand{\divrg}[0]{\nabla\dotp} %
\newcommand{\curl}[0]{\nabla\times} %

\title{Math 418 Notes}
\author{Henry Xia}
%\date{15 September 2017}

\begin{document}

\maketitle

\tableofcontents

%%%%%%%%%%%%%%%%%%%%%%%%%%%%%%%%%%%%%%%%%%%%%%%%%%%%%%%%%%%%%%%%%%%%%%
%%%%%%%%%%%%%%%%%%%%%%%%%%%%%%%%%%%%%%%%%%%%%%%%%%%%%%%%%%%%%%%%%%%%%%
% 2018 09 05

\section{Introduction}

\textbf{Text}: R. Durrett. Probability Theory and Examples. Version 5a.

\begin{defn}
    A \emph{sample space} $\Omega$ is a non-empty set.

    The set of subsets $\mathcal{F} \in 2^\Omega$ is a \emph{field} if and only if
    \begin{enumerate}
        \item[(i)] $\emptyset\in\mathcal{F}$.
        \item[(ii)] Closed under complement. $\forall A \in \mathcal{F}$, $A^C \in \mathcal{F}$.
        \item[(iii)] Closed under finite union. $\forall A,B \in \mathcal{F}$, $A\cup B \in \mathcal{F}$.
    \end{enumerate}
\end{defn}

\begin{defn}
    The set of subsets $\mathcal{F} \in 2^\Omega$ is a \emph{$\sigma$-field} iff 
    \begin{enumerate}
        \item[(i)] $\emptyset\in\mathcal{F}$.
        \item[(ii)] Closed under complement. $\forall A \in \mathcal{F}$, $A^C \in \mathcal{F}$.
        \item[(iii)] Closed under countable union.
    \end{enumerate}
\end{defn}

\begin{defn}
    Call $(\Omega, \mathcal{F})$ a \emph{measureable space} if $\mathcal{F}$ is a $\sigma$-field.
\end{defn}

\begin{defn}
    We write $A = \dot\bigcup_{i\in{I}} A_i$ if the sets $A_i$ are disjoint.
\end{defn}

\begin{defn}
    Let $(\Omega,\mathcal{F})$ be a measurable space. The \emph{measure} $\mu$ on
    $(\Omega,\mathcal{F})$ is a countably additive, non-negative set function
    $\mu:\mathcal{F}\to[0,\infty]$ such that 
    \begin{enumerate}
        \item[(i)] $\mu(\emptyset) = 0$
        \item[(ii)] $\mu(\dot\bigcup_{i\in{I}} A_i) = \sum_{i\in{I}} \mu(A_i)$
    \end{enumerate}
    where $I = \NN$ or $I = \{1,\dots,n\}$.

    $(\Omega,\mathcal{F},\mu)$ is a \emph{measure space}. If $\mu$ is a probability, that is
    $\mu(\Omega)=1$, then the said measure space is a \emph{probability space}.
\end{defn}

\begin{defn}
    A \emph{finite additive measure} on $(\Omega,\mathcal{F})$, where $\mathcal{F}$ is a field, is a
    function $\mu:\mathcal{F}\to[0,\infty]$ such that
    \begin{enumerate}
        \item[(i)] $\mu(\emptyset) = 0$
        \item[(ii)] $\mu(\dot\bigcup_{i\in{I}} A_i) = \sum_{i\in{I}} \mu(A_i)$
    \end{enumerate}
    where (ii) only holds for $I$ finite.
\end{defn}

\begin{example}
    An example of a finite additive measure is
    \[
        \mu(A) = \begin{cases}
            0 \text{ if } $A$ \text{ is finite } \\
            \infty \text{ if } $A$ \text{ is infinite }
        \end{cases}
    \]
\end{example}

\begin{lemma}
    Let $\mathcal{F}$ be a field and $\mu$ be a finite additive measure, then 
    \begin{enumerate}
        \item[(a)] Monotonicity. $A,B\in\mathcal{F}$ with $A\subset B$ implies $\mu(A)\le\mu(B)$.
        \item[(b)] Subadditivity. $A,B\in\mathcal{F}$ means $\mu(A\cup B) \le \mu(A)+\mu(B)$. 
    \end{enumerate}
\end{lemma}
\proof[Proof of (a).] Elementary. \qedhere
\proof[Proof of (b).] Use (a) and disjoint union. \qedhere


%%%%%%%%%%%%%%%%%%%%%%%%%%%%%%%%%%%%%%%%%%%%%%%%%%%%%%%%%%%%%%%%%%%%%%
%%%%%%%%%%%%%%%%%%%%%%%%%%%%%%%%%%%%%%%%%%%%%%%%%%%%%%%%%%%%%%%%%%%%%%
% 2018 09 07

\begin{defn}
    We write $A_n \uparrow A$ if $A_n\subset A_{n+1}$ and $\bigcup A_n = A$.
    Defined similarly for $\downarrow$.
\end{defn}

\begin{thm}
    Let $\mu$ be a measure on $(\Omega,\m{F})$, and $\set{A_n:n\in\NN}$ be a sequence in $\m{F}$.
    \begin{enumerate}
        \item[(a)] Continuity from below. $A_n\uparrow A \implies \mu(A_n) \uparrow \mu(A)$
        \item[(b)] Continuity from above. $\mu(A_1)<\infty \text{ and } A_n\downarrow A \implies
            \mu(A_n) \downarrow \mu(A)$
        \item[(c)] Countable subadditivity. $\mu(\bigcup_n^\infty A_n) \le \sum_n^\infty \mu(A_n)$
    \end{enumerate}
\end{thm}

\proof[Proof of (a)]
Observe that $A_n = \dot\bigcup_{k=1}^n (A_k\sm A_{k-1})$ where we define $A_0 = \emptyset$. Then
$$ \mu(A) = \sum_{k=1}^\infty \mu(A_k\sm A_{k-1}) = \lim_{n\to\infty} \mu(A_n) . $$
Also, $\mu(A_n) \le \mu(A_{n+1})$, so $\mu(A_n)\uparrow\mu(A)$. 
\qedhere

\proof[Proof of (b)]
We can take complements with respect to $A_1$ and use (a) on $(A_1 \sm A_n) \uparrow (A_1 \sm A)$.
Then $\mu(A_1 \sm A_n) \uparrow \mu(A_1 \sm A)$. 
\qedhere

\proof[Proof of (c)]
We use finite subadditivity as follows
\[
    \mu(\cup_k^n A_k) \le \sum_k^n \mu(A_k) \le \lim_{n\to\infty} \sum_k^n \mu(A_k). \qedhere
\]
\qedhere

Partial converse to Theorem 1.2 (Continuity from below): If $\mu$ is finitely additive on
$(\Omega,\m{F})$ and continuous from below, then $\mu$ is a measure (countably additive).


\begin{example}
    Consider the counting measure. Let $A_n = \set{n,n+1,\dots}$, then $A_n\downarrow\emptyset$, but
    $\mu(A_n) = \infty$ for all $n$ and $\mu(\emptyset) = 0$.
\end{example}

\begin{defn}
    A $\m{A}\subset 2^\Omega$ is a $\pi$-system if it is closed under intersection.

    A $\m{A}\subset 2^\Omega$ is a $\lambda$-system if
    \begin{itemize}
        \item[(i)] $\Omega\in\m{A}$
        \item[(ii)] If $A,B\in\m{A}$ and $B\subset A$, then $A\sm B \in \m{A}$
        \item[(iii)] If $A_n\in\m{A}$ and $A_n \uparrow A$, then $A\in\m{A}$
    \end{itemize}
\end{defn}

\begin{thm}[$\pi$-$\lambda$ Theorem]
    Let $\m{C}$ be a $\pi$-system in $2^\Omega$, and $\m{L}$ be a $\lambda$-system in $2^\Omega$. If
    $\m{C}\subset\m{L}$, then $\sigma(\m{C})\subset\m{L}$.
\end{thm}
\proof
Draw a venn diagram, or show that the smallest $\lambda$-system containing $\m{C}$ is a
$\sigma$-field.
\qedhere

%%%%%%%%%%%%%%%%%%%%%%%%%%%%%%%%%%%%%%%%%%%%%%%%%%%%%%%%%%%%%%%%%%%%%%
%%%%%%%%%%%%%%%%%%%%%%%%%%%%%%%%%%%%%%%%%%%%%%%%%%%%%%%%%%%%%%%%%%%%%%
% 2018 09 12

\begin{cor}
    Let $P_1$ and $P_2$ be probabilities on $(\Omega,\m{F})$, and $\m{C}\subset\m{F}$ be a
    $\pi$-system. If $P_1=P_2$ on $\m{C}$, then $P_1=P_2$ on $\sigma(\m{C})$.
\end{cor}
\proof
Let $\m{L} = \set{A\in\m{F}: P_1(A) = P_2(A)}$, then $\m{C}\subset\m{L}$. We want to show that
$\m{L}$ is a $\lambda$-system. 
\begin{enumerate}
    \item[(i)] Clearly $\Omega \in \m{L}$. 
    \item[(ii)] If $A,B\in\m{L}$ and $A\subset B$, then $B\sm A \in \m{L}$ by finite additivity.
    \item[(iii)] If $A_n\in\m{L}$ and $A_n\uparrow A$, then we can use continuity from below to get
        $P_1(A) = P_2(A)$. 
\qedhere
\end{enumerate}
\qedhere

\begin{example}
    Let $p_1,\dots,p_6 \in [0,1]$ and $\sum_{i=1}^6 p_i = 1$. We can define $\Omega =
    \set{1,\dots,6}$, $\m{F} = 2^\Omega$, and $P(\set{i}) = p_i$. Then by finite additivity, we get
    $P(A) = P(\dot\cup_{i\in A} \set{i}) = \sum_{i\in A} p_i$ for any $A\in\m{F}$.
\end{example}


% SECTION
\section{The Law of Large Numbers}

\subsection{The Law of Large Numbers for Coin Tossing}

\textbf{Experiment}: Toss a fair coin an infinite number of times.

\textbf{Outcome}: An infinite sequence of 0's and 1's. $w = (w_1,w_2,\dots)$.

Let $\Omega = \set{0,1}^\NN$ be the sample space. 
Let $x_n:\Omega\to\set{0,1}$ with $x_n(w) = w_n$, and $s_n:\Omega\to\set{0,1,\dots,n}$ with $s_n(w)
= \sum_{i=1}^n x_i(w)$. 

\begin{defn}
    The event $A\subset\Omega$ is a finite dimensional event if $\exists n, B\subset\set{0,1}^n$
    such that $A = \set{w:(w_1,w_2,\dots,w_n)\in B}$.  
    (ie. we only need to look at the first $n$ things.)

    Let $\m{F}_0$ be the set of finite dimensional events.
\end{defn}

We want to show that $\lim_{n\to\infty} \frac{s_n(n)}{n} = \frac12$.

\begin{lemma}
    $\m{F}_0$ is a field.
\end{lemma}
\proof
Observe that $\emptyset = \set{w\in\Omega:w_1\in\emptyset}\in\m{F}_0$.

Let $A_1, A_@ \in \m{F}$, then there exist $n_1, n_2, B_1, B_2$ such that $A_i =
\set{w:(w_1,\dots,w_{n_i})\in B_i}$ for $i\in\set{1,2}$. Now we may assume without loss of
generality that $n_1=n_2$, because we can extend the smaller $B_i$ so that it has the same
dimension as the larger one. This is also closed under complement. 
\qedhere

\begin{defn}
    Define $P:\m{F}_0\to[0,1]$ such that $P(\set{w:(w_1,\dots,w_n)\in B})=\frac{\#B}{2^n}$,
    where $B\subset\set{0,1}^n$. We can check that $P$ is well defined.
\end{defn}

\begin{lemma}
    $P$ is a probability.
\end{lemma}
\proof
We can check that $P(\emptyset) = 0$ and $P(\Omega) = 1$.

Let disjoint $A_1, A_2 \in \m{F}$. Then $\emptyset = A_1\cap A_2 =
\set{w:(w_1,\dots,w_n)\in B_1\cap B_2}$, implying that $B_1\cap B_2 = \emptyset$. Then, we
can write the disjoint union in the same way. 
\qedhere

We can compute $P(\set{s_n(w)=k}) = \binom{n}{k} 2^{-n}$, and say ``$s_n$ has binomial
distribution with parameters $(n,\frac12)$''.

Let $C = \set{w:\lim_{n\to\infty} \frac{s_n(w)}{n} = \frac12}$. Observe that
$C\notin\m{F}_0$. However, we will extend $P$ from $\m{F}_0$ to $\sigma(\m{F}_0)$.

%%%%%%%%%%%%%%%%%%%%%%%%%%%%%%%%%%%%%%%%%%%%%%%%%%%%%%%%%%%%%%%%%%%%%%
%%%%%%%%%%%%%%%%%%%%%%%%%%%%%%%%%%%%%%%%%%%%%%%%%%%%%%%%%%%%%%%%%%%%%%
% 2018 09 14

\begin{lemma}
    $C\in\sigma(\m{F}_0)$.
\end{lemma}
\proof
Observe that
\begin{align*}
	w \in C &\iff \lim_{n\to\infty} \frac{S_n(w)}{n} = \frac12 \\
	&\iff \forall M \in \NN,~ \exists N \text{ such that } \forall n\ge N,~
	\abs{\frac{S_n(w)}{n}-\frac12} < \frac1M \\
	&\iff w \in \bigcap_{M=1}^\infty \bigcup_{N=1}^\infty \bigcap_{n=N}^\infty
	\set{w:\abs{\frac{S_n(w)}{n}-\frac12}<\frac1M}
\end{align*}
Let $C_{n,M}$ be the latter set, which is finite dimensional, hence $C_{n,M}\in\m{F}_0$.
It follows that $C\in\sigma(\m{F}_0)$ as $\sigma$-fields are closed under finite unions
and intersections. 
\qedhere

Suppose that we can extend $P$ to $\sigma(\m{F}_0)$ (by Carath\'eodory). Now we need to prove
that $P(C)=1$, that is the event happens \emph{almost surely}.

\begin{lemma}
$P(C_{n,M}^c) \le \frac{M^2}{4n}$.
\end{lemma}
\proof
We observe that $S_n$ is a binomial random variable, and $\frac{n}{2}$ is the mean. Then
we can use Chebychev's inequality.
\[
	P\left(\abs{\frac{S_n}{n}-\frac12}\ge\frac1M\right) 
	= P\left(\abs{S_n-\frac{n}{2}}\ge\frac{n}{M}\right)
	\le \frac{\text{var}(S_n)}{(n/M)^2} = \frac{M^2}{4n} .
\]
There is also a direct calculation proof given on Perkins's webpage.
\qedhere

\begin{thm}[The Strong Law of Large Numbers for Coin Tossing]
	$P(C)=1$. 
\end{thm}
\proof
Let $\hat{C} = \set{w:\lim_{m\to\infty}\frac{S_{m^2}}{m^2}=\frac12}$. This set is equal to
$C$ as shown in homework 1. Similarly, we can write 
\[
	\hat{C} = \bigcap_{M=1}^\infty \bigcup_{N=1}^\infty \bigcap_{m=N}^\infty
	\set{\abs{\frac{S_{m^2}}{m^2}-\frac12}<\frac1M} .
\]
We want to show that $P\left(\bigcap_{m=N}^\infty C_{m^2,M}\right) = 1$, after which
continuity from below and continuity from above will yield the result. 
\begin{align*}
	P\left(\bigcap_{m=N}^\infty C_{m^2,M}\right) 
	&= 1 - P\left(\bigcup_{m=N}^\infty C_{m^2,M}^c\right)
	\ge 1 - \sum_{m=N}^\infty P(C_{m^2,M}^c)
	\ge 1 - \sum_{m=N}^\infty \frac{M^2}{4m^2} \to 1 ,
\end{align*}
where the convergence holds because $\frac1{m^2}$ is summable.

Increasing $N$ means we take the intersection of less sets, which increases.  Continuity
from below gives $P(\bigcup\bigcap{C_{m^2,M}})=1$. 

Increasing $M$ makes the epsilon smaller, so we are decreasing the size of each set. This
will decrease the intersection.  Continuity from above gives
$P(\bigcap\bigcup\bigcap{C_{m^2,M}})=1$. 

Hence $P(C)=P(\hat{C})=1$. 
\qedhere


%%%%%%%%%%%%%%%%%%%%%%%%%%%%%%%%%%%%%%%%%%%%%%%%%%%%%%%%%%%%%%%%%%%%%%
%%%%%%%%%%%%%%%%%%%%%%%%%%%%%%%%%%%%%%%%%%%%%%%%%%%%%%%%%%%%%%%%%%%%%%
% 2018 09 17

Now we prove that we can indeed extend $P$ to $\sigma(\m{F}_0)$. 

\begin{thm}
    Let $\m{C}$ be a $\pi$-system, $\mu_1, \mu_2$ be measures on $\sigma(\m{C})$. If
    $\mu_1=\mu_2$ on $\m{C}$ and both are $\sigma$-finite on $\m{C}$, then $\mu_1=\mu_2$ on
    $\sigma(\m{C})$. 
\end{thm}
\proof
Let $\set{A_n}\in\m{C}$ and $A_n\uparrow\Omega$ where $\mu_1(A_n)=\mu_2(A_n)<\infty$. For
all $\mu_i(A_n) > 0$, define $P_{i,n}(\m{E}) = \frac{\mu_i(E\cap A_n)}{\mu_i(A_n)}$. Then
$P_{i,n}$ is a probability. We can check that $\mu_i(A_n)>0$ for $n$ sufficiently large. We
can also show that $P_{1,n}=P_{2,n}$ using continuity from below. Now we can use
$\pi$-$\lambda$ theorem to finish the proof. 
\qedhere

\begin{lemma}
    Let $P:\m{F}_0\to[0,1]$ be a finitely additive probability, then $P$ is a countably
    additive probability, where $\m{F}_0$ is defined as above.
\end{lemma}
\proof
Give $\Omega=\set{0,1}^\NN$ the product topology. That is for $\set{w_n}\subset\Omega$,
then $w^n\to w$ as $n\to\infty$ if and only if $\forall i\in\NN$, we have
$\lim_{n\to\infty} w_{n,i}=w_i$. Equivalently, we can define the metric $d(w,w') =
\sum_{i=1}^\infty \frac{\abs{w_i-w'_i}}{2^i}$. Observe that $(\Omega,d)$ is compact by
Cantor diagonalization. 

Let $\set{A_n}$ be disjoint sets in $\m{F}_0$ such that $\bigcup_{n=1}^\infty
A_n\in\m{F}_0$. Let $K_n=\bigcup_{j=1}^\infty A_j \sm \bigcup_{j=1}^n A_j$. Observe that
$K_n\downarrow\emptyset$. If there exists some $N$ such that $K_n=\emptyset$ for all $n>N$,
then finite additivity implies countable additivity. 

Suppose (for contradiction) that $K_n\neq\emptyset$ for all $n$. Then ...
\qedhere













\end{document}
