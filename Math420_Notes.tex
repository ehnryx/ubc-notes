\documentclass[11pt]{article}
\usepackage{hyperref}
\usepackage[margin=1in]{geometry}
\usepackage{amsmath}
\usepackage{amsthm}
\usepackage{amssymb}
\usepackage{amsfonts}
\usepackage{graphicx} %\usepackage[pdftex]{graphicx}
\usepackage{xcolor}
\usepackage{setspace}
%\usepackage{tikz}


\setlength\parindent{0pt}

\newtheorem{thm}{Theorem}[section]
\newtheorem{cor}{Corollary}[thm]
\newtheorem{lemma}[thm]{Lemma}
\theoremstyle{definition}
\newtheorem{defn}{Definition}[section]
\newtheorem{example}{Example}[section]
\newtheorem{prop}{Proposition}[section]
\newtheorem{pty}{Property}[section]
\newtheorem{remark}{Remark}[section]
\newtheorem{obs}{Observation}[section]
\newcommand{\The}[2]{\begin{#1}#2\end{#1}}

\newcommand{\ord}[0]{\text{ord}}

% notes
\iftrue 
\newcommand{\f}[2]{\frac{#1}{#2}}
\newcommand{\re}[1]{\frac{1}{#1}}
\newcommand{\half}[0]{\frac{1}{2}}
\newcommand{\ift}[0]{It follows that}
\newcommand{\cp}[1]{\overline{#1}}
\newcommand{\Note}[0]{\noindent\textbf{Note: }} 
\newcommand{\Claim}[0]{\noindent\textbf{Claim: }} 
\newcommand{\Lemma}[1]{\noindent\textbf{Lemma #1}: } %
\newcommand{\Ex}[0]{\noindent\textbf{Example: }} %
\newcommand{\Special}[0]{\noindent\textbf{Special case: }} %
\newcommand{\solution}[2]{\item[]\proof[Solution to #1] #2 \qedhere}
\newcommand{\legendre}[2]{\left(\frac{#1}{#2}\right)}
\newcommand{\dent}[0]{\hspace{0.5in}}
\fi

\newcommand{\sm}[0]{\setminus}
\newcommand{\set}[1]{\left\{ #1 \right\}}
\newcommand{\nl}[0]{\vspace{12pt}}
\newcommand{\rng}[2]{#1,\dots,#2}
\newcommand{\srng}[3]{#1_#2,\dots,#1_#3}
\newcommand{\st}[0]{\text{ such that }}
\newcommand{\et}[0]{\text{ and }}
\newcommand{\then}[0]{\text{ then }}
\newcommand{\forsome}[0]{\text{ for some }}
\newcommand{\floor}[1]{\lfloor #1 \rfloor}

% misc
\newcommand{\abs}[1]{\left\lvert#1\right\rvert} %
% lcm ???
\DeclareMathOperator{\lcm}{lcm} 
% blackboard bold
\newcommand{\RR}{\mathbb{R}}
\newcommand{\FF}{\mathbb{R}}
\newcommand{\QQ}{\mathbb{Q}}
\newcommand{\ZZ}{\mathbb{Z}}
\newcommand{\NN}{\mathbb{N}}
\newcommand{\CC}{\mathbb{C}}
% mathcal
\newcommand{\m}[1]{\mathcal{#1}}
% vectors
\newcommand{\vvec}[1]{\textbf{#1}} %
\newcommand{\ii}[0]{\vvec{i}} %
\newcommand{\jj}[0]{\vvec{j}} %
\newcommand{\kk}[0]{\vvec{k}} %
\newcommand{\hvec}[1]{\hat{\textbf{#1}}} %
\newcommand{\cvec}[3]{ %column vector
    \ensuremath{\left(\begin{array}{c}#1\\#2\\#3\end{array}\right)}}
\newcommand{\pfrac}[2]{\frac{\partial#1}{\partial#2}} %
\newcommand{\norm}[1]{\left\lVert#1\right\rVert} %
% dotp roduct
\makeatletter
\newcommand*\dotp{\mathpalette\dotp@{.5}}
\newcommand*\dotp@[2]{\mathbin{\vcenter{\hbox{\scalebox{#2}{$\m@th#1\bullet$}}}}}
\makeatother
% divrg and curl
\newcommand{\divrg}[0]{\nabla\dotp} %
\newcommand{\curl}[0]{\nabla\times} %

\title{Math 420 Notes}
\author{Henry Xia}
%\date{15 September 2017}

\begin{document}

\maketitle

\tableofcontents

%%%%%%%%%%%%%%%%%%%%%%%%%%%%%%%%%%%%%%%%%%%%%%%%%%%%%%%%%%%%%%%%%%%%%%
%%%%%%%%%%%%%%%%%%%%%%%%%%%%%%%%%%%%%%%%%%%%%%%%%%%%%%%%%%%%%%%%%%%%%%
% 2018 09 07

\section{Introduction}

\begin{defn}
    A $\sigma$-algebra on $X$ is a collection of subsets of $2^X$ that is closed
    under complement and countable union.
\end{defn}

\begin{defn}
    Let $\m{M}\subset 2^X$ be the measurable subsets of $X$. A measure $\mu$ on
    $(X,\m{M})$ is a function $\mu:\m{M}\to[0,\infty]$ satisfying the following.
    \begin{enumerate}
        \item[(i)] $\mu(\emptyset) = 0$
        \item[(ii)] $\mu(\dot\bigcup_j^\infty E_j) = \sum_j^\infty \mu(E_j)$
    \end{enumerate}
    Note that $\m{M}$ is a $\sigma$-algebra.
\end{defn}

\begin{example}
    The counting measure. $\mu(E) = \#\set{X:X\in E}$
\end{example}
\begin{example}
    The Dirac measure. Fix $x_0\in X$. $\mu(E) = 1$ if $x_0\in E$, and $\mu(E) =
0$ otherwise.  \end{example}
\begin{example}
    An unmeasurable set. (Folland p.20).

    Let $E_r = E+r$ mod 1. There exists a set $E\subset[0,1)$ such that 
    \begin{itemize}
        \item $\set{E_r}_{r\in\QQ\cap[0,1)}$ are disjoint
        \item $\bigcup_{r\in\QQ\cap[0,1)} E_r = [0,1)$
    \end{itemize}
    This set $E$ is inconsistent with (ii) of the definition when $\mu([0,1)) =
    1$ and $\mu(E_r) = \mu(E)$.
\end{example}

\begin{defn}
    Let non-empty $\m{E}\subset 2^X$. The $\sigma$-algebra generated by $\m{E}$
    is $\m{M}(\m{E})$, that is the smallest $\sigma$-algebra containing $\m{E}$.
    (We can get this by taking the intersection of all the $\sigma$-algebras
    containing $\m{E}$)
\end{defn}

\begin{example}
    Let $X$ be a topological space. The Borel $\sigma$-algebra $B_X$ is a
    $\sigma$-algebra generated by open sets. This contains open sets, closed
    sets, countable union of closed sets ($F\sigma$-sets), countable
    intersection of open sets ($G\delta$-sets).

    $B_\RR$ can be generated by any of
    \begin{itemize}
        \item open intervals. $\set{(a,b)}$
        \item closed intervals. $\set{[a,b]}$
        \item half open intervals. $\set{(a,b]}$
        \item semi-infinite intervals. $\set{(a,\infty)}$
    \end{itemize}
\end{example}

%%%%%%%%%%%%%%%%%%%%%%%%%%%%%%%%%%%%%%%%%%%%%%%%%%%%%%%%%%%%%%%%%%%%%%
%%%%%%%%%%%%%%%%%%%%%%%%%%%%%%%%%%%%%%%%%%%%%%%%%%%%%%%%%%%%%%%%%%%%%%
% 2018 09 12

\section{The Lebesgue Measure}

\subsection{Premeasures}
Let $\m{A}$ be the set of finite disjoint unions of h-intervals, where
h-intervals are of the following form: $(a,b]$, $(a,\infty)$, $\emptyset$, where
$-\infty\le a<b<\infty$. 

\begin{prop}
    $\m{A}$ is an algebra.
\end{prop}
\proof
The intersection of two h-intervals is also an h-interval. The complement of an
h-interval is the union of at most two disjoint h-intervals. Refer to text
(Folland Prop 1.7).
\qedhere

Define the ``Length'' of sets in $\m{A}$ to be a function
$m_0:\m{A}\to[0,\infty]$ with finite additivity and $m_0(\emptyset)=0$.

\begin{defn}
    A premeasure is a function $m:\m{A}\to[0,\infty]$ such that
    \begin{enumerate}
        \item[(i)] $m(\emptyset) = 0$
        \item[(ii)] For countably many disjoint $A_j\in\m{A}$ whose union
            $A=\bigcup A_j$ is also in $\m{A}$, we have
            $m(\bigcup{A_j})=\sum{m(A_j)}$.
    \end{enumerate}
\end{defn}

\begin{thm}
    The following is true
    \begin{enumerate}
        \item $m_0$ is well defined.
        \item $m_0$ is a premeasure.
    \end{enumerate}
\end{thm}

\proof[Proof of 1.]
This is just bookkeeping. See text.
\qedhere

\proof[Proof of 2.]
Let $A=(a,b]\in\m{A}$ be a countable disjoint union of $A_j=(a_j,b_j]\in\m{A}$. We can
assume that $A_j$, because each $A_j$ would otherwise be the finite union of some disjoint
set of intervals in $\m{A}$. We can also assume that $A$ is an interval by the same
argument. 

Consider $A = \bigcup_{j=1}^n A_j \cup (A\sm\bigcup_{j=1}^n)$. Then we have 
\[
    m_0(A) = m_0\left(\bigcup_{j=1}^n A_j\right) + m_0\left(A\sm\bigcup_{j=1}^n\right)
    \ge m_0\left(\bigcup_{j=1}^n A_j\right) .
\]
Taking the limit gives $m_0(A) \ge m_0(\bigcup_{j=1}^\infty A_j)$.

Now let $\epsilon > 0$. Consider the compact interval $[a+\epsilon,b]$ covered by
$\bigcup_{j=1}^\infty (a_j, b_j+\frac{\epsilon}{2^j})$. There must be a finite subcover.
Now, $(a+\epsilon,b]$ is also covered by this finite subcover, and we can relabel the
finite subcover so that $a_j<a_{j+1}$. Then 
\begin{align*}
    m_0(A) &= m_0((a,a+\epsilon]) + m_0((a+\epsilon,b]) \\
            &\le \epsilon + m_0((a_1,b_n+\frac{\epsilon}{2^n}))
            = \epsilon + b_n+\frac{\epsilon}{2^n} - a_n + \sum_{j=2}^n(a_j-a_{j-1}) \\
            &\le \epsilon + (b_n-a_n) + \sum_j^n\left(b_j+\frac{\epsilon}{2^j}-a_{j-1}\right)
            \le \epsilon + \sum_j^n\frac{\epsilon}{2^j} + \sum_j^n m_0(A_j) \\
            &\le 7\epsilon + \sum m_0(A_j) ,
\end{align*}
and countable additivity follows.
\qedhere


%%%%%%%%%%%%%%%%%%%%%%%%%%%%%%%%%%%%%%%%%%%%%%%%%%%%%%%%%%%%%%%%%%%%%%
%%%%%%%%%%%%%%%%%%%%%%%%%%%%%%%%%%%%%%%%%%%%%%%%%%%%%%%%%%%%%%%%%%%%%%
% 2018 09 14

\subsection{Lebesgue Outer Measure}

\begin{defn}
    The Lebesgue outer measure $m^*$ of a set $E\subset\RR$ is defined as follows. 
    \[
        m^*(E) = \inf\set{\sum_{j=1}^\infty m_0(I_j):E\subset\bigcup_{j=1}^\infty I_j}.
    \]
\end{defn}

\begin{defn}
    In general, given $\m{E}\subset2^X$ with $\emptyset,X\in\m{E}$ and
    $\mu_0:\m{E}\to[0,\infty]$ with $\mu_0(\emptyset)=0$, we can define
    $\mu^*:2^X\to[0\infty]$ as follows.
    \[
        \mu^*(E) = \inf\set{\sum_{j=1}^\infty \mu_0(E_j):E\subset\bigcup_{j=1}^\infty E_j
        \text{ and } E_j\in\m{E}}.
    \]
\end{defn}

\begin{prop}
    $\mu^*$ is an outer measure, where an outer measure satisfies three properties.
    \begin{enumerate}
        \item $\mu^*(\emptyset)=0$
        \item $A\subset B \implies \mu^*(A)\le\mu^*(B)$
        \item $\mu^*\left(\bigcup_{j=1}^\infty E_j\right) \le \sum_{j=1}^\infty\mu^*(E_j)$
    \end{enumerate}
\end{prop}

\proof[Proof of 1.]
$\emptyset\subset\bigcup_{j=1}^\infty\emptyset \implies
\mu^*(\emptyset)\le\sum_{j=1}^\infty\mu_0(\emptyset)=0$.
\qedhere

\proof[Proof of 2.]
Let $A\subset B$. Then $\set{\set{E_j}_j\subset\m{E}:B\subset\bigcup_j E_j} \subset
\set{\set{E_j}_j\subset\m{E}:A\subset\bigcup_j E_j}$. Hence $\mu^*(A)\le\mu^*(B)$. 
\qedhere

\proof[Proof of 3.]
Let $\set{A_{j,k}}_k \subset \m{E}$ such that $E_j \subset \bigcup_{k=1}^\infty A_{j,k}$.
Observe that $\bigcup_{j=1}^\infty E_j \subset \bigcup_{j,k=1}^\infty A_{j,k}$. Let
$\epsilon > 0$, then
\[
    \mu^*\left(\bigcup_{j=1}^\infty E_j\right)
    \le \sum_{j,k=1}^\infty \mu_0(A_{j,k})
    \le \sum_{j=1}^\infty \left(\frac\epsilon{2^j}+\mu^*(E_j)\right)
    = \epsilon + \sum_{j=1}^\infty \mu^*(E_j)
\]
Since $\epsilon$ is arbitrary, we get $\mu^*(\bigcup_j E_j) \le \sum_j \mu^*(E_j)$. 
\qedhere

Observe that $\mu^*$ is defined for every set in $2^X$, but it is not additive. To fix
this, we will remove some ``bad'' sets.

\begin{defn}
    Let $\mu^*$ be an outer measure on $X$.  A set $A\subset X$ is $\mu^*$-measurable if
    for every $E\subset X$, we have $\mu^*(E) = \mu^*(E\cap A) + \mu^*(E\cap A^c)$. 
\end{defn}


%%%%%%%%%%%%%%%%%%%%%%%%%%%%%%%%%%%%%%%%%%%%%%%%%%%%%%%%%%%%%%%%%%%%%%
%%%%%%%%%%%%%%%%%%%%%%%%%%%%%%%%%%%%%%%%%%%%%%%%%%%%%%%%%%%%%%%%%%%%%%
% 2018 09 17

\subsection{Carath\'eodory's Extension Theorem}

\begin{thm}
    Let $\m{M}$ be the set of $\mu^*$-measurable sets, and $\mu^*\upharpoonright_\m{M}$ is
    a complete measure. 
\end{thm}

\proof
\begin{enumerate}
    \item We show that $\m{M}$ is an algebra. Clearly, $\emptyset\in\m{M}$, and $\m{M}$ is
        closed under complement. Now let $A,B\in\m{M}$. Then 
        \begin{align*}
            \mu^*(E) &= \mu^*(E\cap A) + \mu^*(E\cap A^c) \\
            &= \mu^*(E\cap A\cap B) + \mu^*(E\cap A\cap B^c) 
            + \mu^*(E\cap A^c\cap B) + \mu^*(E\cap A^c\cap B^c) \\
            &\ge \mu^*(E\cap(A\cup B)) + \mu^*(E\cap(A\cup B)^c) .
        \end{align*}
        The other inequality is automatic by monoticity. Hence $A\cup B \in \m{M}$. 

    \item We show that $\mu^*$ is finitely additive. Let $A,B\in\m{M}$ be disjoint. Then
        \[
            \mu^*(A\cup B) = \mu^*((A\cup B)\cap A) + \mu^*((A\cup B)\cap A^c)
            = \mu^*(A) + \mu^*(B) .
        \]

    \item We show that $\m{M}$ is closed under countable union and $\mu^*$ is countably
        additive.

        Let $\set{A_j} \subset \m{M}$, $B_n = \bigcup_{j=1}^n A_j$, and $B =
        \bigcup_{j=1}^\infty A_j$. Let $E\subset X$, then 
        \begin{align*}
            \mu^*(E\cap B_n) &= \mu^*(E\cap B_n\cap A_n) + \mu^*(E\cap B_n\cap A_n^c) \\
            &= \mu^*(E\cap A_n) + \mu^*(E\cap B_{n-1}) = \sum_{j=1}^n \mu^*(E\cap A_j) .
        \end{align*}
        By the definition, we get 
        \[
            \mu^*(E) = \mu^*(E\cap B_n) + \mu^*(E\cap B_n^c)
            \ge \sum_{j=1}^n\mu^*(E\cap A_j) + \mu^*(E\cap B^c) .
        \]
        Take $n\to\infty$, then we get closure under countable union.
        \[
            \mu^*(E)\ge\sum_{j=1}^\infty\mu^*(E\cap A_j) + \mu^*(E\cap B^c)
            \ge \mu^*(E\cap B) + \mu^*(E\cap B^c) .
        \]
        Take $E=B$, then we get countable additivity.
        \[
            \mu^*(B) = \sum_{j=1}^\infty \mu^*(A_j) + \mu^*(\emptyset)
            = \sum_{j=1}^\infty \mu^*(A_j) .
        \]
\end{enumerate}
We can easily check that $\m{M}$ is complete. This theorem is complete.
\qedhere


%%%%%%%%%%%%%%%%%%%%%%%%%%%%%%%%%%%%%%%%%%%%%%%%%%%%%%%%%%%%%%%%%%%%%%
%%%%%%%%%%%%%%%%%%%%%%%%%%%%%%%%%%%%%%%%%%%%%%%%%%%%%%%%%%%%%%%%%%%%%%
% 2018 09 19

\begin{prop}
  If $A\in\m{A}$, then $A$ is $\mu^*$-measurable.
\end{prop}
\proof
Let $A\in\m{A}$ and $E\subset X$. Let $\epsilon>0$. There exists $\set{A_j}\subset\m{A}$
with $E\subset \bigcup_{j=1}^\infty A_j$ such that $\mu^*(E)+\epsilon\ge\sum_{j=1}^\infty
\mu_0(A_j)$ by the definition of $\mu^*$. Then
\[
  \mu^*(E)+\epsilon \ge \sum\mu_0(A_j\cap A)+\sum\mu_0(A_j\cap A^c)
  \ge \mu^*(E\cap A) + \mu^*(E\cap A^c) .
\]
Take $\epsilon\to0$, and we see that $A$ is $\mu^*$-measurable.
\qedhere

\begin{prop}
  $\mu^*\upharpoonright_\m{A} = \mu_0\upharpoonright_\m{A}$.
\end{prop}
\proof See text.  \qedhere


\subsection{Lebesgue Measure on $\RR$}

Let $X=\RR$ and define $m_0$ to be the length of h-intervals. 
\begin{enumerate}
  \item $m^*(E)=\inf\set{\sum m_0(I_j) : E\subset\bigcup I_j}$, where $I_j$ are
  h-intervals. 
  \item $\m{L}$ is the $m^*$-measurable sets (Lebesgue measurable). 
  \item $m=m^*\upharpoonright_\m{L}$. 
\end{enumerate}

\begin{remark}
  The measure $m$ is a Borel measrue, that is it is defined for all Borel sets. Also, $m$
  is the unique Borel measure with $m((a,b])=b-a$. 
\end{remark}
\proof
See text. Basically if $\mu_0$ is $\sigma$-finite on $\m{A}$, then
Carath\'eodory gives uniqueness. 
\qedhere

\begin{remark}
We can also construct a measure with any non-decreasing right-continuous $F:\RR\to\RR$ with
$m_F((A,b])=F(b)-F(a)$. This is the Lebesgue-Stieltjes measure. Observe that the Lebesgue
measure simply has $F(x)=x$. 
\end{remark}

\begin{prop}
Any Boren measure $\mu$ that is finite on bounded sets defines a non-decreasing
right-continuous function $F:\RR\to\RR$ as follows
\[
  F(x)=\begin{cases}
    \mu((0,x])  &\text{ if } x>0 \\
    0           &\text{ if } x=0 \\
    \mu((-x,0]) &\text{ if } x<0
  \end{cases} .
\]
\end{prop}

\begin{prop}
The Lebesgue measure is \textbf{translation invariant} $m(E+s)=m(E)$ and \textbf{dilation
invariant} $m(rE)=\abs{r}m(E)$.
\end{prop}


%%%%%%%%%%%%%%%%%%%%%%%%%%%%%%%%%%%%%%%%%%%%%%%%%%%%%%%%%%%%%%%%%%%%%%
%%%%%%%%%%%%%%%%%%%%%%%%%%%%%%%%%%%%%%%%%%%%%%%%%%%%%%%%%%%%%%%%%%%%%%
% 2018 09 21

\begin{remark}
  $\text{open sets, closed sets, etc.}\subsetneq\m{B}_\RR\subsetneq\m{L}\subsetneq2^\RR$. 
\end{remark}

\begin{lemma}
  $m^*(E)=\inf\set{\sum_{j=1}^\infty(b_j-a_j):E\subset\bigcup_{j=1}^\infty(a_j,b_j)}$. 
\end{lemma}
\proof See text. \qedhere

\begin{thm}
  Let $E\subset\RR$. All of the following imply one another. 
  \begin{enumerate}
    \item[(a)] $E\in\m{L}$.
    \item[(b)] There exists $U_\text{open}\supset E$ such that $m^*(U\sm E)\le\epsilon$. 
    \item[(c)] There exists $F_\text{closed}\subset E$ such that $m^*(E\sm F)\le\epsilon$. 
    \item[(d)] There exists a G$\delta$ set $V\supset E$ such that $E=V\sm N_1$ with $N_1$
    null. 
    \item[(e)] There exists a F$\sigma$ set $H\supset E$ such that $E=H\cup N_2$ with $N_2$
    null. 
  \end{enumerate}
\end{thm}

\proof .
\begin{itemize}
\item[] $(a\implies b)$. Let $\epsilon>0$. There exists $U=\bigcup_{j=1}^\infty I_j \supset E$
(where each $I_j$ is an open interval) such that $m(E)+\epsilon\ge\sum_{j=1}^\infty m(I_j)
\ge m(U)$. Then, using the definition of a measurable set,
\[
  m(U) = m(U\cap E) + m(U\cap E^c) = m(E) + m(U\cap E^c) \le m(E)+\epsilon .
\]
Hence $\m(U\sm E)<\epsilon$ holds for $m(E)$ finite.

If $m(E)$ is infinite, then let $E_j=E\cap(j,j+1]$ and $U_j=U\cap(j,j+1]$. Then $m(U_j\sm
E_j) \le \epsilon2^{-\abs{j}}$ from the finite case, and countable additivity gives the
desired result.

\item[] $(a\implies c)$. Use $E^c$ and (a) implies (b).

\item[] $(b\implies d)$. There exists open $U_j\supset E$ such that $m^*(U_j\sm
E)\le\frac1j$. Then $V=\bigcap_{j=1}^\infty U_j \supset E$ is a G$\delta$ set. Let
$N_1=V\sm E$, then $E=V\sm N_1$. It follows that $N_1\subset U_j\sm E$ for all $j$, so
$m^*(N_1)\le m^*(U_j\sm E)\le \frac1j$. Hence $m^*(N_1)=0$, so $N_1$ is a null set.

\item[] $(c\implies e)$. Similar to above. 

\item[] $(d\implies a)$ and $(e\implies a)$. G$\delta$ and F$\sigma$ sets are Borel, so
they are measurable. Null sets are also measurable by completeness. Hence $E$ is
measurable. 
\end{itemize}
\qedhere



%%%%%%%%%%%%%%%%%%%%%%%%%%%%%%%%%%%%%%%%%%%%%%%%%%%%%%%%%%%%%%%%%%%%%%
%%%%%%%%%%%%%%%%%%%%%%%%%%%%%%%%%%%%%%%%%%%%%%%%%%%%%%%%%%%%%%%%%%%%%%
% 2018 09 24

\section{Integrals and Convergence}

\begin{defn}
  Let $(X,\m{M})$ and $Y,\m{N})$ be measurable spaces. A functoin $f:X\to Y$ is
  $(\m{M},\m{N})$ measurable if $f^{-1}(E)\in\m{M}$ for all $E\in\m{N}$. 
\end{defn}

\begin{remark}
  If $\m{N}$ is generated by $\m{E}\subset\m{N}$, then $f:X\to Y$ is measurable if and only
  if $f^{-1}(E)\in\m{M}$ for all $E\in\m{E}$.
\end{remark}

\begin{remark}
  A function $f:\RR\to\RR$ is Borel measurable if it is continuous.
\end{remark}

\begin{remark}
  Composition of measurable functions is measurable. 
\end{remark}

\begin{remark}
  $f:X\to\bar\RR$ is measurable if it is $(\m{M},\m{B}_{\bar\RR})$-measurable, where
  $$\m{B}_{\bar\RR}=\set{E\subset\bar\RR:E\cap\RR\in\m{B}_{\bar\RR}}.$$
\end{remark}

\begin{prop}
  Let $(X,\m{M})$ be a measurable space. Then 
  \begin{enumerate}
    \item $f:X\to\CC$ is measurable if and only if the real and imaginary parts of $f$ are
    measurable.
    \item $f,g:X\to\CC$ are measurable implies $f+g$ and $f\cdot g$ are measurable.
    \item $f_j:X\to\bar\RR$ is measurable implies $\sup_jf_j$, $\inf_jf_j$,
    $\limsup_{j\to\infty}f_j$ and $\liminf_{j\to\infty}f_j$ are measurable. 
    \item $f_j:X\to\CC$ is measurable implies $\lim_{j\to\infty}f_j$ is measurable if the
    limit exists.
  \end{enumerate}
\end{prop}



%%%%%%%%%%%%%%%%%%%%%%%%%%%%%%%%%%%%%%%%%%%%%%%%%%%%%%%%%%%%%%%%%%%%%%
%%%%%%%%%%%%%%%%%%%%%%%%%%%%%%%%%%%%%%%%%%%%%%%%%%%%%%%%%%%%%%%%%%%%%%
% 2018 09 26

\begin{defn}
  A simple function on $(X,\m{M})$ is of the form $f(x)=\sum_{j=1}^nz_j\chi_{E_j}(x)$ for
  $z_j\in\CC$ and $E_j\in\m{M}$. 
\end{defn}

\begin{remark}
  $f$ is in ``standard form'' if $E_j=f^{-1}(\set{z_j})$. 
\end{remark}

\begin{defn}
  Let $(X,\m{M},\mu)$ be a measure space and let $f=\sum_{j=1}^nz_j\chi_{E_j}$ be a simple
  function. Then 
  \[
    \int f = \sum_{j=1}^nz_j\mu(E_j) .
  \]
\end{defn}

\begin{prop}
  Let $\phi,\psi$ be simple functions. 
  \begin{enumerate}
    \item[(a)] $c\in\CC$ implies $\int c\psi = c\int\psi$. (linearity)
    \item[(b)] $\int\phi+\psi=\int\phi+\int\psi$. 
    \item[(c)] If $\phi,\psi\in\RR$, then $\phi\le\psi\implies\int{\phi}\le\int{\psi}$. 
    \item[(d)] If $\phi\ge0$, then $\m{M}\ni A\mapsto\int_A\phi:=\int\chi_A\phi$ is a
    measure.
  \end{enumerate}
\end{prop}

\proof .
\begin{enumerate}
  \item[] {\color{red} (a), (b), and (d). See text / exercise. }
  \item[] (c). Let $\phi=\sum_{j=1}^nz_j\chi_{E_j}$ and $\psi=\sum_{k=1}^mw_kX_{F_k}$ in
  standard form. Then 
  \begin{align*}
    \int\phi &= \sum_jz_j\mu(E_j) = \sum_jz_j\sum_k\mu(E_j\cap F_k) =
    \sum_j\sum_kz_j\mu(E_j\cap F_k) \\
    &\le \sum_k\sum_j w_k\mu(E_j\cap F_k)
    = \sum_kw_k\sum_j\mu(E_i\cap F_k) = \sum_kw_k\mu(F_k) = \int\psi .
  \end{align*}
\end{enumerate}
\qedhere

\begin{defn}
  Define $L^+=\set{f:X\to[0,\infty),\text{measurable}}$. Then for $f\in L^+$, define
  \[
    \int f = \sup\set{\int\phi : 0\le\phi\le f, ~\phi\text{ simple}} . 
  \]
\end{defn}

\begin{remark}
  We have monotonicity and linearity for $f\in L^+$. 
\end{remark}


%%%%%%%%%%%%%%%%%%%%%%%%%%%%%%%%%%%%%%%%%%%%%%%%%%%%%%%%%%%%%%%%%%%%%%
%%%%%%%%%%%%%%%%%%%%%%%%%%%%%%%%%%%%%%%%%%%%%%%%%%%%%%%%%%%%%%%%%%%%%%
% 2018 09 28

\subsection{Approximation by Simple Functions and Monotone Convergence}

\begin{thm}[Approximation Theorem]
  \begin{enumerate}
    \item[(a)] Let measurable $f:X\to[0,\infty]$. There exists simple
    $0\le\phi_1\le\phi_2\le\cdots\le f$ such that $\phi_n\to f$ pointwise, and $\phi_n\to
    f$ uniformly on sets where $f$ is bounded. 
    \item[(b)] Let measurable $f:X\to\CC$. There exists simple $\set{\phi_n}$ with
    $0\le\abs{\phi_1}\le\abs{\phi_2}\le\cdots\le\abs{f}$ such that $\phi_n\to f$ pointwise,
    and $\phi_n\to f$ uniformly on sets where $f$ is bounded. 
  \end{enumerate}
\end{thm}
\proof
Proof by construction with powers of 2. 
\qedhere

\begin{thm}[Monotone Convergence  Theorem]
  Let $\set{f_n}\subset L^+$ with $0\le f_1\le f_2\cdots$. Then 
  \[
    \int\lim_{n\to\infty}f_n = \lim_{n\to\infty}\int f_n .
  \]
\end{thm}
\proof
  Let $f(x)=\sup_nf_n(x)\in L^+$. Then $\set{\int f_n}$ is increasing, so
  $\lim_{n\to\infty}\int f_n = \sup_n\int f_n$ (which exists). Since $f_n\le f$, we have
  $\int f_n\le \int f$, so $\lim_{n\to\infty}\int f_n\le \int f$.

  Let $\phi$ be a simple function such that $0\le\phi\le f$. Fix $\alpha\in(0,1)$. Let
  $E_n=\set{x:f_n(x)\ge\alpha\phi(x)}$. Observe that $E_n$ is measurable and $E_1\subset
  E_2\subset \cdots \bigcup_{n=1}^\infty E_n = X$.  Since $E\mapsto\int_E\phi$ is a
  measure, we get $\int_{E_n}\phi\mapsto\int\phi$ by continuity from below. Then
  \[
    \int f_n \ge \int_{E_n}f_n \ge \alpha\int_{E_n}\phi
    \implies \lim_{n\to\infty} f_n \ge \alpha\int\phi .
  \]
  If we take $\alpha\to1$, then $\lim\int f_n \ge \int\phi$. Then the Monotone Convergence
  Theorem follows by simple function approximation.
\qedhere

\begin{prop}
  Let $\set{f_n}\subset L^+$, then $\int\sum_nf_n=\sum_n\int f_n$.
\end{prop}
\proof
Let $f_1,f_2\in L^+$. By approximation, we have $\phi_n\uparrow f_1$ and $\psi_n\uparrow
f_2$. Then $\phi_n+\psi_n\uparrow f_1+f_2$. Then by Monotone Convergence, 
\[
  \int f_1+f_2 = \lim_{n\to\infty}\int(\phi_n+\psi_n) =
  \lim_{n\to\infty}\left(\int\phi_n+\int\psi_n\right) =
  \lim_{n\to\infty}\int\phi_n+\lim_{n\to\infty}\int\psi_n = \int f_1 + \int f_2 .
\]
Now let $\set{f_n}_{n=1}^\infty$. Then using MCT on
$\sum_{n=1}^Nf_n\uparrow\sum_{n=1}^\infty f_n$ implies 
\[
  \int\sum_{n=1}^\infty f_n = \int\lim_{N\to\infty}\sum_{n=1}^Nf_n
  = \lim_{N\to\infty}\int\sum_{n=1}^Nf_n = \lim_{N\to\infty}\sum_{n=1}^N\int f_n
  = \sum_{n=1}^\infty\int f_n .
\]
\qedhere


%%%%%%%%%%%%%%%%%%%%%%%%%%%%%%%%%%%%%%%%%%%%%%%%%%%%%%%%%%%%%%%%%%%%%%
%%%%%%%%%%%%%%%%%%%%%%%%%%%%%%%%%%%%%%%%%%%%%%%%%%%%%%%%%%%%%%%%%%%%%%
% 2018 10 01

\begin{prop}
  If $f\in L^+$, then $\int f=0 \iff f=0$ almost everywhere. 
\end{prop}
\proof
For simple $f=\sum_{k=1}^na_k\chi_{E_k}$, then $\mu(E_k)=0$ or $a_k=0$. The result follows
because the finite union of null sets is still a null set.

Now we prove this for $f\in L^+$. If $f=0$ almost everywhere, then
any simple $\phi$ satisfying $0\le\phi\le f$ is also 0 almost everywhere. Then
$\int\phi=0$, implying that $\int f=0$. If $f$ is not 0 almost everywhere, then
$\mu(\set{f(x)>0})>0$. Let $E_n=\set{f(x)>\frac1n}$ for $n=\in\NN$. Then
$\set{f(x)>0}=\bigcup E_n$. It follows that there exists some $k$ such that $\mu(E_k)>0$.
Hence $f\ge\frac1k\chi_{E_k}$, so $\int f \ge \frac1k\mu(E_k) > 0$.
\qedhere

\begin{remark}
  We don't care about null sets. If $f_n\in L^+$ and $f_n\uparrow f$ almost everywhere,
  then $\int f = \lim_{n\to\infty}\int f_n$. 
\end{remark}
\proof
Apply MCT to $f_n\chi_{N^c}$ where $N$ is the null set on which $f_n$ does not converge to
$f$. 
\qedhere

\begin{thm}[Fatou's Lemma]
  Let $\set{f_n}_{n=1}^\infty\subset L^+$. Then 
  \[
    \int\liminf_{n\to\infty}f_n \le \liminf_{n\to\infty}\int f_n .
  \]
\end{thm}
\begin{cor}
  If $f_n\to f$ almost everywhere, then $\int f \le \liminf_{n\to\infty} \int f_n$. 
\end{cor}

\proof[Proof of Fatou's Lemma]
Let $g_k(x)=\inf_{n\ge k}f_n(x)$ be an increasing sequence of functions. 
Then for each $j\ge k$, we have
\[
  \inf_{n\ge k}f_n \le f_j \implies \int\inf_{n\ge k}f_n \le \int f_j
  \implies \int\inf_{n\ge k}f_n \le \inf_{j\ge k}\int f_j
\]
It follows by MCT and the above that
\[
  \int\sup_kg_k = \int\liminf f_n = \lim_{k\to\infty}\int\inf_{n\ge k}f_n \\
  \le \lim_{k\to\infty}\inf_{j\ge k}\int f_j = \liminf_{n\to\infty}\int f_n .
\]
\qedhere

%%%%%%%%%%%%%%%%%%%%%%%%%%%%%%%%%%%%%%%%%%%%%%%%%%%%%%%%%%%%%%%%%%%%%%
%%%%%%%%%%%%%%%%%%%%%%%%%%%%%%%%%%%%%%%%%%%%%%%%%%%%%%%%%%%%%%%%%%%%%%
% 2018 10 03

\subsection{Integration of Complex Functions}

Let $f:X\to\RR$ be measurable, then $f^+=\max(f,0)$ and $f^-=\max(-f,0)$. Observe that
$f^+,f^-\in L^+$ and we can write $f=f^+-f^-$. Also observe that if $f:X\to\CC$, then we
write $f=\Re(f)+i\Im(f)$, and hence $\int f = \int\Re(f)+i\int\Im(f)$.

\begin{defn}
  We say $f:X\to\CC$ is integrable if $\int\abs{f}<\infty$. Define
  \[
    L^1(\mu)=\set{f:X\to\CC:\int\abs{f}<\infty} .
  \]
\end{defn}

\begin{prop}
  \begin{enumerate}
    \item[(a)] $L^1$ is a vector space.
    \item[(b)] $\int$ is a linear map on $L^1$. 
    \item[(c)] $f\in L^1$ implies $\abs{\int f}\le\int\abs{f}$. 
    \item[(d)] If $f,g\in L^1$, then $\int\abs{f-g}=0\iff f=g\text{ a.e.}\iff
    \int_Ef=\int_Eg$ for all $E\in\m{M}$. 
  \end{enumerate}
\end{prop}
\proof See text.  \qedhere

\begin{remark}
  If we define $L^1$ to be the equivalence class of almost everywhere defined integrable
  functions under $f\sim g \iff f=g$ a.e., then $L^1$ is a Banach space under $\abs{f-g}$. 
\end{remark}

\begin{remark}
  If $f\in L^+$ with $\int f<\infty$, then $\mu(\set{f=\infty})=0$. 
\end{remark}
\proof {\color{red}Exercise.} \qedhere

\begin{thm}[Dominated Convergence Theorem]
  Let $L^1\ni f_n\to f$ almost everywhere and $\abs{f_n}\le g\in L^1$ for all $n$. Then
  $f\in L^1$ and $\int f = \lim_{n\to\infty}\int f_n$. 
\end{thm}
\proof
First we show that $f\in L^1$. Observe that $\abs{f_n}\le g$ implies $\abs{f}\le g$ almost
everywhere, so $f\in L^1$. 

We take $f_n\in\RR$. Otherwise, consider $\Re(f_n)$ and $\Im(f_n)$. Observe that $g\pm
f_n\ge 0$, so Fatou's Lemma implies
\begin{align*}
  \int g + \int f &= \int g+f \le \liminf\int g+f_n = \int g + \liminf\int f_n \\
  \int g - \int f &= \int g-f \le \liminf\int g-f_n = \int g - \limsup\int f_n
\end{align*}
Since $\int g<\infty$, we have $\limsup\int f_n\le \int f \le \liminf\int f_n$. It follows
that $\lim\int f_n = \int f$. 
\qedhere

\begin{prop}
  Let $\set{f_j}_{j=1}^\infty\subset L^1$ with $\sum_{j=1}^\infty\int\abs{f_j}<\infty$.
  Then $\sum_{j=1}^\infty f_j$ converges almost everywhere and $\int\sum_{j=1}^\infty f_j =
  \sum_{j=1}^\infty\int f_j$.
\end{prop}
\proof
Each $\abs{f_j}\in L^+$, so MCT gives $\int\sum_{j=1}^\infty\abs{f_j} =
\sum_{j=1}^\infty\int\abs{f_j}<\infty$. Hence, $\sum_{j=1}^\infty\abs{f_j}\in L^1$. It
follows that $\sum_{j=1}^\infty\abs{f_j(x)}<\infty$ almost everywhere, so
$\sum_{j=1}^\infty f_j$ converges almost everywhere. Since
$\abs{\sum_{j=1}^Nf_j}\le\sum_{j=1}^N\abs{f_j}\le\sum_{j=1}^\infty\abs{f_j}=g\in L^1$, we
can apply DCT to the partial sums to get the result.
\qedhere

\begin{defn}
  The support of a function $f:X\to\CC$ is the set $\set{x:f(x)\neq0}$.
\end{defn}

\begin{thm}[$L^1$ Approximation of Functions]
  Let $f\in L^1(\mu)$. For any $\epsilon>0$, there exists a simple functoin
  $\phi=\sum_{j=1}^na_j\chi_{E_j}$ such that $\int\abs{f-\phi}<\epsilon$.

  If $(X,\mu)=(\RR,m)$, then we can take each $E_j$ to be a finite union of open intervals.
  Also, there exists a continuous function $g$ with compact support such that
  $\int\abs{f-g}<\epsilon$. 
\end{thm}

%%%%%%%%%%%%%%%%%%%%%%%%%%%%%%%%%%%%%%%%%%%%%%%%%%%%%%%%%%%%%%%%%%%%%%
%%%%%%%%%%%%%%%%%%%%%%%%%%%%%%%%%%%%%%%%%%%%%%%%%%%%%%%%%%%%%%%%%%%%%%
% 2018 10 05

% THE DATES ARE OFF BUT THAT'S FINE




%%%%%%%%%%%%%%%%%%%%%%%%%%%%%%%%%%%%%%%%%%%%%%%%%%%%%%%%%%%%%%%%%%%%%%
%%%%%%%%%%%%%%%%%%%%%%%%%%%%%%%%%%%%%%%%%%%%%%%%%%%%%%%%%%%%%%%%%%%%%%
% 2018 10 08

\section{Modes of Convergence}

Let $f_n:X\to\CC$ and $f:X\to\CC$. 

\begin{defn}
  $f_n\to f$ pointwise if $f_n(x)\to f(x)$ for all $x\in X$. 
\end{defn}

\begin{defn}
  $f_n\to f$ uniformly if $\sum_{x\in X}\abs{f_n(x)-f(x)}\to0$.
\end{defn}

\begin{defn}
  $f_n\to f$ almost everywhere if $f_n(x)\to f(x)$ for all $x\in N^c$ with $\mu(N)=0$.
\end{defn}

\begin{defn}
  $f_n\to f$ in $L^1$ if $\int_X\abs{f_n-f}d\mu\to0$. 
\end{defn}

\begin{defn}
  $f_n\to f$ in measure if for all $\epsilon>0$,
  $\mu\left(\set{x:\abs{f_n(x)-f(x)}\ge\epsilon}\right)\to0$. 
\end{defn}


%%%%%%%%%%%%%%%%%%%%%%%%%%%%%%%%%%%%%%%%%%%%%%%%%%%%%%%%%%%%%%%%%%%%%%
%%%%%%%%%%%%%%%%%%%%%%%%%%%%%%%%%%%%%%%%%%%%%%%%%%%%%%%%%%%%%%%%%%%%%%
% 2018 10 10

We have the following implications
\begin{itemize}
  \item uniform convergence implies pointwise convergence
  \item pointwise convergence implies almost everywhere convergence
  \item convergence in $L^1$ implies convergence in measure
  \item uniform convergence implies convergence in measure
  \item uniform convergence implies convergence in $L^1$ on a finite measure space
  \item almost everywhere convergence implies convergence in measure on a finite measure
  space
  \item almost everywhere convergence implies convergence in $L^1$ if we can apply DCT
  \item convergence in measure implies almost everywhere convergence if we allow
  subsequences
\end{itemize}

\begin{thm}[Egoroff]
  If $\mu(X)<\infty$, and $\set{f_n}_{n=1}^\infty$ are measurable, with $f_n\to f$ almost
  everywhere, then $f_n\to f$ almost uniformly. That is, for any $\epsilon>0$, there exists
  $E\subset X$ with $\mu(E)<\epsilon$ such that $f_n\to f$ uniformly on $E^c$.
\end{thm}
\begin{remark}
  Almost uniform convergence implies convergence in measure.
\end{remark}
\proof
  For $k\in\NN$ let $E_n(k)=\bigcup_{m=n}^\infty\set{x\in X:\abs{f_m(x)-f(x)}\ge\frac1k}$.
  These sets are decreasing with $\mu\left(\bigcap_{n=1}^\infty E_n(k)\right)=0$ by almost
  everywhere convergence. By continuity from above, we have
  $\lim_{n\to\infty}\mu(E_n(k))=0$. Hence, for any $k$ and $\epsilon$, there is some $n_k$
  such that $\mu(E_{n_k}(k))<\epsilon2^{-k}$, so
  \[
    \mu\left(E:=\bigcup_{k=1}^\infty E_{n_k}(k)\right) \le \sum_{k=1}^\infty\mu(E_{n_k}(k))
    < \epsilon .
  \]
  If $x\notin E$, then $\abs{f_n(x)-f(x)}<\frac1k$ for sufficiently large $n$, so $f_n\to
  f$ uniformly on $E^c$.
\qedhere

\begin{defn}
  A sequence of functions $f_n$ is Cauchy in measure if for any $\epsilon>0$,
  \[
    \lim_{m,n\to\infty}\mu(\set{\abs{f_n-f_m}\ge\epsilon})=0 .
  \]
\end{defn}

\begin{thm}
  Let $\set{f_n}_{n=1}^\infty$ be Cauchy in measure. Then 
  \begin{itemize}
    \item $f_n\to f$ in measure for some $f$. 
    \item There exists a subsequence $f_{n_j}$ that converges to $f$ almost everywhere. 
    \item If $f_n\to g$ in measure, then $f=g$ almost everywhere. 
  \end{itemize}
\end{thm}
\proof See text. \qedhere




%%%%%%%%%%%%%%%%%%%%%%%%%%%%%%%%%%%%%%%%%%%%%%%%%%%%%%%%%%%%%%%%%%%%%%
%%%%%%%%%%%%%%%%%%%%%%%%%%%%%%%%%%%%%%%%%%%%%%%%%%%%%%%%%%%%%%%%%%%%%%
% 2018 10 12

\section{Product Measures}

Consider measure spaces $(X,\m{M},\mu)$ and $(Y,\m{N},\nu)$. 

\begin{defn}
  Define $\m{M}\otimes\m{N}$ to be the $\sigma$-algebra generated by rectangles of the form
  $A\times B = \set{(x,y):x\in A, y\in B}$ where $A\in\m{M}$ and $B\in\m{N}$.
\end{defn}

Let $\m{A}$ be the set of finite disjoint unions of rectangles. Then
$\pi:\m{A}\to[0,\infty]$ with $\bigcup(A_j\times B_j)\mapsto\sum\mu(A_j)\nu(B_j)$ is a
well-defined premeasure.

\begin{defn}
  The product measure $\mu\times\nu$ is the extension of $\pi$ to $\m{M}\otimes\m{N}$.
\end{defn}

\begin{defn}
  Let $E\in X\times Y$. Define $E_x=\set{y\in Y:(x,y)\in E}$, and $E^y=\set{x\in X:(x,y)\in
  E}$.

  Let $f:X\times Y\to\CC$. Define $f_x:y\mapsto f(x,y)$ and $f^y:x\mapsto f(x,y)$.
\end{defn}

\begin{prop}
  Let $E\in\m{M}\otimes\m{N}$. Then $E_x\in\m{N}$ and $E^y\in\m{M}$. 
\end{prop}
\proof
  Let $R=\set{E\subset X\times Y:E_x\in\m{N},~E^y\in\m{M}}$. Then $R$ contains all the
  rectangles. Furthermore, $R$ is a $\sigma$-algebra (exercise). Hence
  $\m{M}\otimes\m{N}\subset R$. 
\qedhere

\begin{prop}
  If $f$ is $\m{M}\otimes\m{N}$-measurable, then $f_x$ is $\m{N}$-measurable and $f^y$ is
  $\m{M}$-measurable. 
\end{prop}
\proof
  $f_x^{-1}(B)=(f^{-1}(B))_x\in\m{N}$ for any Borel $B$. 
\qedhere



%%%%%%%%%%%%%%%%%%%%%%%%%%%%%%%%%%%%%%%%%%%%%%%%%%%%%%%%%%%%%%%%%%%%%%
%%%%%%%%%%%%%%%%%%%%%%%%%%%%%%%%%%%%%%%%%%%%%%%%%%%%%%%%%%%%%%%%%%%%%%
% 2018 10 15

\begin{thm}[Slicing Theorem]
  Let $E\in\m{M}\otimes\m{N}$. Then $x\mapsto\nu(E_x)$ and $y\mapsto\mu(E^y)$ are
  measurable, and $(\mu\times\nu)(E)=\int_X\nu(E_x)d\mu(x)=\int_Y\mu(E^y)d\nu(y)$.
\end{thm}

\proof
Suppose that $\mu$ and $\nu$ are finite.
\begin{enumerate}
  \item We check that the conclusion holds for rectangles $E=A\times B$. Observe that
  \[
    \nu(E_x)=\chi_A(x)\nu(B) \implies \int_X\nu(E_x)d\mu=\mu(A)\nu(B)=(\mu\times\nu)(E).
  \]

  \item The conclusion also holds for finite disjoint unions of rectangles by the
  additivity of measures and integrals. 

  \item Let $\m{C}=\set{E\in\m{M}\otimes\m{N}:\text{ conclusion holds}}$. Then
  $\m{A}\subset\m{C}$ where $\m{A}$ is the finite disjoint union of rectangles. 

  We show that $\m{C}$ is a monotone class. That is it is closed under increasing union and
  decreasing intersection (ie. monotone union and intersection). 
  \begin{itemize}
    \item[$\bigcup$:] Let increasing $\set{E_n}\subset\m{C}$ with $E=\bigcup E_n$. Then 
    $f_n(y)=\mu(E_n^y)\uparrow f(y)=\mu(E^y)$. Hence by MCT and continuity from below, 
    \[
      \int_Y\mu(E^y)d\nu = \lim_{n\to\infty}\int_Y\mu(E_n^y)d\nu =
      \lim_{n\to\infty}(\mu\times\nu)(E_n) = (\mu\times\nu)(E) .
    \]
    Therefore, $E\in\m{C}$. 
    \item[$\bigcap$:] Let decreasing $\set{E_n}\subset\m{C}$ with $E=\bigcap E_n$. Then
    $f_n(y)=\mu(E_n^y)\le f_1(y)<\infty$ by monotonicity and finiteness of $\mu$ and $\nu$.
    Hence by DCT and continuity from above, 
    \[
      \int_Y\mu(E^y)d\nu = \lim_{n\to\infty}\int_Y\mu(E_n^y)d\nu =
      \lim_{n\to\infty}(\mu\times\nu)(E_n) = (\mu\times\nu)(E) .
    \]
    Therefore, $E\in\m{C}$. 
  \end{itemize}
  Since $\m{C}$ is a monotone class that contains $\m{A}$, it contains the $\sigma$-algebra
  generated by $\m{A}$ (See text for proof). Hence $\m{C}=\m{M}\otimes\m{N}$.

  \item If $\mu$ and $\nu$ are $\sigma$-finite, then consider $X\times Y =
  \bigcup_{j=1}^\infty A_j\times B_j$ as the union of rectangles with finite measure. Then
  apply the above to each rectangle. Details omitted. 
\end{enumerate}
\qedhere


%%%%%%%%%%%%%%%%%%%%%%%%%%%%%%%%%%%%%%%%%%%%%%%%%%%%%%%%%%%%%%%%%%%%%%
%%%%%%%%%%%%%%%%%%%%%%%%%%%%%%%%%%%%%%%%%%%%%%%%%%%%%%%%%%%%%%%%%%%%%%
% 2018 10 17

\begin{thm}[Tonelli]
Let $f\in L^+(\mu\times\nu)$ with $\sigma$-finite $\mu$ and $\nu$. Then
$x\mapsto\int_Yf_x~d\nu\in L^+(\mu)$ and $y\mapsto\int_Xf^y~d\mu\in L^+(\nu)$, and
\[
  \int_X\int_Yf~d\nu d\mu = \int_Y\int_Xf~d\mu d\nu = \int_{X\times Y}f~d(\mu\times\nu) . 
\]
\end{thm}
\proof
  \begin{itemize}
    \item If $f=\chi_E$, use the slicing theorem. 
    \item If $f\in L^+$ is simple, follows from above by additivity.
    \item If $f\in L^+$, let $0\le f_n^\text{simple}\uparrow f$. Let
    $g_n(x)=\int_Y(f_n)_x~d\nu$ and $g(x)=\int_Yf_x~d\nu$. Then $g_n\uparrow g$ and $\int
    g_n \to \int g$ by MCT, and we get the result
    \[
      \int_{X\times Y}f~d(\mu\times\nu) = \lim_{n\to\infty}\int_{X\times
      Y}f_n~d(\mu\times\nu) = \lim_{n\to\infty}\int_Xg_n~d\mu = \int_Xg~d\mu .
    \]
  \end{itemize}
\qedhere

\begin{thm}[Fubini]
  Let $f\in L^1(\mu\times\nu)$. Then $f_x\in L^1(\nu)$ a.e.$x$, $f^y\in L^1(\mu)$ a.e.$y$,
  $x\mapsto\int_Yf_x~d\nu\in L^1(\mu)$, and $y\mapsto\int_Xf^y~d\mu\in L^1(\nu)$. 

  ... TLDR: We can also change the order of iterated integrals. 
\end{thm}
\proof
  Let $f:X\times Y\to\CC$. Write $f=(\Re(f)^+-\Re(f)^-)+i(\Im(f)^+-\Im(f)^-)$ and apply
  Tonelli. 
\qedhere

\subsection{Lebesgue Measure on $\RR^n$}
All the nice properties for $\RR$ also happen in $\RR^n$ ( + invariance of rotation ). 


%%%%%%%%%%%%%%%%%%%%%%%%%%%%%%%%%%%%%%%%%%%%%%%%%%%%%%%%%%%%%%%%%%%%%%
%%%%%%%%%%%%%%%%%%%%%%%%%%%%%%%%%%%%%%%%%%%%%%%%%%%%%%%%%%%%%%%%%%%%%%
% 2018 10 19

% DATES GOT SCREWED UP AGAIN RIP



%%%%%%%%%%%%%%%%%%%%%%%%%%%%%%%%%%%%%%%%%%%%%%%%%%%%%%%%%%%%%%%%%%%%%%
%%%%%%%%%%%%%%%%%%%%%%%%%%%%%%%%%%%%%%%%%%%%%%%%%%%%%%%%%%%%%%%%%%%%%%
% 2018 10 22

\section{Differentiation of Measures}

\begin{example}
  Let $(X,\m{M},\mu)$ be a measurable space, and $f\in L^+$. Let $\nu:\m{M}\to[0,\infty]$
  such that $\nu:E\mapsto\int_Ef~d\mu$. Observe that $\nu$ is a measure, and we write
  $d\nu=f~d\mu$. 
\end{example}

\begin{defn}
  Define an extended $\mu$-integrable function as follows.  Let $g:X\to[-\infty,\infty]$
  be measurable. We can write $g=g^+-g^-$ with $g^+,g^-\in L+$ such that we have
  $\int_Xg^+~d\mu<\infty$ or $\int_Xg^-~d\mu<\infty$. 
\end{defn}

Observe that if $g$ is extended $\mu$-integrable, and we have a function
$\nu:E\mapsto\int_Eg~d\mu$, then at most one of $\infty$ and $-\infty$ can be in the range
of $\nu$. 

\begin{defn}
  A signed measure on $(X,\m{M})$ is a function $\nu:\m{M}\to[-\infty,\infty]$ such that
  \begin{itemize}
    \item $\nu(\emptyset)=0$
    \item $\nu$ assumes at most one of $\infty$ and $-\infty$
    \item If $\set{E_j}_{j=1}^\infty\subset\m{M}$ are disjoint, then
    $\nu\left(\bigcup_{j=1}^\infty E_j\right) = \sum_{j=1}^\infty\nu(E_j)$
  \end{itemize}
\end{defn}

\begin{defn}
  The set $E\in\m{M}$ is positive, negative, or null for $\nu$ if respectively
  $\nu(F)\ge0$, $\nu(F)\le0$, or $\nu(F)=0$ for all measurable $F\subset E$.
\end{defn}

\begin{example}
  Let $\nu:E\mapsto\int_Eg~d\mu$. Then $E$ is positive for $\nu$ if and only if $g\ge0$
  $\mu$ almost everywhere ($\mu$-a.e.) on $E$. 
\end{example}

\begin{thm}[Hahn Decomposition]
  Let $\nu$ be a signed measure on $(X,\m{M})$. There exists sets $P$ positive for $\nu$
  and $N$ negative for $\nu$ such that $X=P\cup N$ with $P\cap N = \emptyset$.

  If there exists another pair of such sets $P'$ and $N'$, then $P\triangle P'$ and
  $N\triangle N$ are null for $\nu$. 
\end{thm}

\proof
Without loss of generality, assume that $\nu:\m{M}\to[-\infty,\infty)$. Otherwise, we can
take $-\nu$. Let $m=\sup\set{\nu(E):E\text{ positive for }\nu}$, so we have positive sets
$P_j$ such that $\lim_{j\to\infty}\nu(P_j)=m$. Let $P=\bigcup_{j=1}^\infty P_j$. Observe
that $P$ is also positive for $\nu$. It follows from continuity that $m=\nu(P)<\infty$.
{\color{red}Exercise: show that continuity from above/below holds for signed measures.}

Let $N=X\sm P$. We show by contradiction that $N$ does not contain any positive subsets.
Suppose that $N$ contains some positive $E$ with $\nu(E)>0$, then $\nu(P\cup E) =
\nu(P)+\nu(E) > m$, which contradicts the maximality of $m$.

Suppose that $N$ contains some set $A$ with $\nu(A)>0$. Since $A$ is not positive,
there exists some $C\subset A$ with $\nu(C)<0$. Then if $B=A\sm C$, we have $B\subset A$
and $\nu(B)=\nu(A)-\nu(C)>\nu(A)$.

Now we show that we can construct some problematic set.  We can define a sequence
$\set{A_j}_{j=1}^\infty$ as follows.
\begin{itemize}
  \item Choose $A_1\subset N$ with $\nu(A_1)>\frac1{n_1}$ where $n_1$ is the smallest
  positive integer for the inequality to hold.
  \item For $j>1$, choose $A_j\subset A_{j-1}$ with $\nu(A_j)>\nu(A_{j-1})+\frac1{n_j}$
  where $n_j$ is the smallest positive integer for the inquality to hold.
\end{itemize}
Let $A=\bigcap_{j=1}^\infty A_j$. Note that $\nu(A)$ needs to be finite, so the sum
$\sum_{j=1}^\infty\frac1{n_j}$ must be summable, which means $n_j\to\infty$.  But, since
$\nu(A)>0$, there is some $B\subset A$ with $\nu(B)>\nu(A)$, that is
$\nu(B)>\nu(A)+\frac1n$ for some sufficiently large $n$. So we contradict the minimality
of $n_j$ when $n_j>n$. This means that $N$ is negative, so we are done. 

Now we consider the uniqueness of $P$ and $N$. Let $X=P'\cup N'$ where $P'$ is positive
and $N'$ is negative, and these two sets are disjoint. Then the symmetric difference is 
\[
  P\triangle P' = (P\cap N') \cup (N\cap P') .
\]
These two intersections are both $\nu$-null, so we are done. 
\qedhere


\begin{defn}
  Signed measures $\mu$ and $\nu$ on $(X,\m{M})$ are mutually singular if there exists
  some disjoint $A,B\in\m{M}$ such that $X=A\cup B$ where $B$ is $\mu$-null and $A$ is
  $\nu$-null. We write $\mu\perp\nu$. 
\end{defn}

\begin{example}
  Consider $(\RR,\m{B}_\RR)$. Then $\delta_{x_0}$ is mutually singular to the Lebesgue
  measure, but not the counting measure. 
\end{example}

\begin{thm}[Jordan Decomposition]
  Let $\nu$ be a signed measure on $(X,\m{M})$. Then there exist unique positive measures
  $\nu^+$ and $\nu^-$ such that $\nu=\nu^+-\nu^-$ and $\nu^+\perp\nu^-$. 
\end{thm}
\proof
Let $X=P\cup N$ be the Hahn decomposition. Let $\nu^+(E)=\nu(E\cap P)$ and
$\nu^-(E)=-\nu(E\cap N)$. Observe the following. 
\begin{itemize}
  \item $(\nu^+-\nu^-)(E)=\nu(E\cap P)+\nu(E\cap P^c)=\nu(E)$ 
  \item $P$ is $\nu^-$-null and $N$ is $\nu^+$-null, so $\nu^+\perp\nu^-$. 
\end{itemize}
For uniqueness, suppose $\nu=\mu^+-\mu^-$ with $\mu^+\perp\mu^-$. Then $X=A\cup B$ with
$\mu^+(B)=\mu^-(A)=0$ for some disjoint $A,B\in\m{M}$. Observe that this implies that
$X=A\cup B$ is Hahn, so $A\triangle P$ is null. Then 
\[
  \mu^+(E)=\mu^+(E\cap A)=\nu(E\cap A)=\nu(E\cap P)=\nu^+(E\cap P)=\nu^+(E) ,
\]
so $\mu^+=\nu^+$. We can do the same for $\mu^-=\nu^-$. 

{\color{red}Exercise: show that $A\triangle P \implies \nu(E\cap A)=\nu(E\cap P)$.}
\qedhere

\begin{defn}
  Let $\abs{\nu}=\nu^++\nu^-$ be the total variation measure of $\nu$.  \end{defn}

{\color{red}
Exercise: $E$ $\nu$-null $\iff$ $\abs{\nu}(E)=0$. 

Exercise: $\nu\perp\mu \iff \abs{\nu}\perp\mu \iff \nu^+\perp\mu \text{ and }
\nu^-\perp\mu$. 
}

\subsection{The Radon-Nikodym Derivative}

Let $f$ be extended $\mu$-integrable and $\nu:E\mapsto\int_Ef~d\mu$ where $\mu$ is a
positive measure. We write $d\nu=f~d\mu$. 
\setstretch{0.8}
\begin{remark} Here are some remarks. 
  \begin{itemize}
    \item $X = P\cup N = \set{f>0}\cup\set{f\le0}$ 
    \item $f=f^+-f^-$, so $\int_Ef^+~d\mu-\int_Ef^-~d\mu=\nu^+-\nu^-$
    \item $\abs{f}=f^++f^-$, so
    $(\nu^++\nu^-)(E)=\abs{\nu}(E)=\int_E(f^++f^-)~d\mu=\int_E\abs{f}~d\mu$. Hence
    $d\abs{\nu}=d\abs{\mu}$
    \item $\nu\perp\mu$ only if $\nu=0$. That is $f=0$ $\mu$-a.e.
    \item $\nu(E)=\int_E(\chi_P-\chi_N)~d\abs{\nu}$ 
    \item Integration with respect to signed $\nu$: $L^1(\nu)=L^1(\nu^+)\cap L^1(\nu^-) =
    L^1(\abs{\nu})$. Hence $\int f~d\nu = \int f~d\nu^+ - \int f~d\nu^-$
    \item $\abs{\nu(E)}\le\abs{\nu}(E)$ (by the triangle inequality)
  \end{itemize}
\end{remark}
\setstretch{1.0}

\begin{defn}
  let $\nu$ be signed and $\mu\ge0$ on $(X,\m{M})$. We say $\nu$ is absolutely continuous
  with respect to $\mu$ if $\nu(E)=0$ for all $E\in\m{M}$ where $\mu(E)=0$. We write
  $\nu\ll\mu$. 
\end{defn}

{\color{red}
Exercise: $\nu\ll\mu \iff \nu^+\ll\mu \text{ and } \nu^-\ll\mu$. 

Exercise: $\nu\ll\mu \text{ and } \nu\perp\mu \implies \nu=0$. 
}

\begin{remark}
  If $d\nu=f~d\mu$, then $\nu\ll\mu$. 
\end{remark}

%%%
\begin{thm}
  If $\nu$ is finite, that is $\abs{\nu}(X)<\infty$, then $\nu\ll\mu$ if and only if
  $\forall\epsilon>0$, $\exists\delta>0$ such that $E\in\m{M}$ with $\mu(E)<\delta$
  implies $\abs{\nu(E)}<\epsilon$.
\end{thm}

\proof
If the latter holds, then the former must hold. (Eg. take a sequence of decreasing sets and
use continuity from above). 

Let $\nu\ll\mu$. Without loss of generality, assume that $\nu\ge0$. Otherwise,
$\abs{\nu(E)}\le\abs{\nu}(E)$, so we consider $\abs{\nu}$ instead. We proceed by
contradiction. There exists some $\epsilon>0$ and for each $j$, we have $E_j\in\m{M}$ such
that $\mu(E_j)<2^{-j}$ and $\nu(E_j)>\epsilon$. Set $F_k=\bigcup_{j=k}^\infty E_j$.
Observe that $F_k\searrow F = \bigcap_{k=1}^\infty F_k$. Then 
\[
  \mu(F_k) \le \sum_{j=k}^\infty\mu(E_j) \le \sum_{j=k}^\infty2^{-j} = \frac1{2^{k-1}} . 
\]
Now by continuity from above, $\mu(F)=\lim_{k\to\infty}\mu(F_k)=0$. Since $E_k\subset
F_k$, we get 
\[
  \nu(F) = \lim_{k\to\infty}\nu(F_k) \ge \lim_{k\to\infty}\nu(E_k) \ge \epsilon .
\]
However, this contradicts $\nu\ll\mu$, so we are done. 
\qedhere


\begin{defn}
  Let $\nu$ be signed and $\mu\ge0$ on $(X,\m{M})$. Then we can write $\nu=\lambda+\rho$
  with $\lambda\perp\mu$ and $\rho\ll\mu$. This is the Lebesgue Decomposition. 
\end{defn}

\begin{thm}[Radon-Nikodym Theorem]
  Let $\nu$ be signed and $\mu\ge0$ on $(X,\m{M})$ where both are $\sigma$-finite. Then
  there exists a unique Lebesgue Decomposition $\nu=\lambda+\rho$, and there exists a
  unique extended $\mu$-integrable $f:X\to\RR$ such that $d\rho=f~d\mu$ (up to $\mu$-null
  sets). Then $\rho(E)=\int_Ef~d\mu$, and $f=\frac{d\rho}{d\mu}$ is the Radon-Nikodym
  derivative of $\rho$ with respect to $\mu$. 
\end{thm}

\proof
We can assume $\nu\ge0$, because we can take $\nu=\nu^+-\nu^-$. 
We will assume that $\nu$ and $\mu$ are finite (see text for an extension to
$\sigma$-finite). 

Let $\m{F}=\set{f\in L^+ : \int_Ef~d\mu \le \nu(E) ~~\forall E\in\m{M}}$. Observe that
$0\in\m{F}$, so $\m{F}$ is non-empty. Observe that if $f,g\in\m{F}$, then 
\[
  \int_E\max(f,g)~d\mu = \int_{E\cap\set{f>g}}f~d\mu+\int_{E\cap\set{f\le g}}g~d\mu
  \le \nu(E\cap\set{f>g}) + \nu(E\cap\set{f\le g}) = \nu(E) ,
\]
so $\max(f,g)\in\m{F}$. 

Let $a=\sup\set{\int f~d\mu : f\in\m{F}}$. Note that $a\le\nu(X)<\infty$ (by assumption of
finiteness), and $a=\lim_{j\to\infty}\int f_j~d\mu$ (by definition of supremum). Set
$g_j=\max(f_1,f_2,...,f_j)\in\m{F}$. Now $\set{g_j}_{j=1}^\infty$ is monotone, so using
MCT, we get 
\[
  a = \lim_{j\to\infty}\int f_j~d\mu \le \lim_{j\to\infty}\int g_j~d\mu \le a ,
\]
so $\int f~d\mu = \lim_{j\to\infty}\int g_j~d\mu = a$. More MCT gives
\[
  \int_Ef~d\mu=\lim_{j\to\infty}\int_Eg_j~d\mu\le\nu(E) ,
\]
so $f\in\m{F}$.

Now set $d\lambda = d\nu - f~d\mu$. 

% TODO ???

\qedhere



% TODO @ class 23










\end{document}
























