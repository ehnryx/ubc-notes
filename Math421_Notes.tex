\documentclass[11pt]{article}
\usepackage{hyperref}
\usepackage[margin=1in]{geometry}
\usepackage{amsmath}
\usepackage{amsthm}
\usepackage{amssymb}
\usepackage{amsfonts}
\usepackage{graphicx} %\usepackage[pdftex]{graphicx}
\usepackage{xcolor}
\usepackage{setspace}
%\usepackage{tikz}


\setlength\parindent{0pt}

\newtheorem{thm}{Theorem}[subsection]
\newtheorem{cor}[thm]{Corollary}
\newtheorem{lemma}[thm]{Lemma}
\theoremstyle{definition}
\newtheorem{defn}[thm]{Definition}
\newtheorem{example}[thm]{Example}
\newtheorem{exe}[thm]{Exercise}
\newtheorem{prop}[thm]{Proposition}
\newtheorem{pty}[thm]{Property}
\newtheorem{remark}[thm]{Remark}
\newtheorem{obs}[thm]{Observation}
\newcommand{\The}[2]{\begin{#1}#2\end{#1}}

\newcommand{\ord}[0]{\text{ord}}

% notes
\iftrue
\newcommand{\f}[2]{\frac{#1}{#2}}
\newcommand{\re}[1]{\frac{1}{#1}}
\newcommand{\half}[0]{\frac{1}{2}}
\newcommand{\ift}[0]{It follows that}
\newcommand{\cp}[1]{\overline{#1}}
\newcommand{\Note}[0]{\noindent\textbf{Note: }}
\newcommand{\Claim}[0]{\noindent\textbf{Claim: }}
\newcommand{\Lemma}[1]{\noindent\textbf{Lemma #1}: } %
\newcommand{\Ex}[0]{\noindent\textbf{Example: }} %
\newcommand{\Special}[0]{\noindent\textbf{Special case: }} %
\newcommand{\solution}[2]{\item[]\proof[Solution to #1] #2 \qedhere}
\newcommand{\legendre}[2]{\left(\frac{#1}{#2}\right)}
\newcommand{\dent}[0]{\hspace{0.5in}}
\fi

\newcommand{\sm}[0]{\setminus}
\newcommand{\set}[1]{\left\{ #1 \right\}}
\newcommand{\nl}[0]{\vspace{12pt}}
\newcommand{\rng}[2]{#1,\dots,#2}
\newcommand{\srng}[3]{#1_#2,\dots,#1_#3}
\newcommand{\st}[0]{\text{ such that }}
\newcommand{\et}[0]{\text{ and }}
\newcommand{\then}[0]{\text{ then }}
\newcommand{\forsome}[0]{\text{ for some }}
\newcommand{\floor}[1]{\lfloor #1 \rfloor}
\newcommand{\s}[0]{\sigma}
\newcommand{\e}[0]{\varepsilon}

% misc
\newcommand{\abs}[1]{\left\lvert#1\right\rvert} %
% lcm ???
\DeclareMathOperator{\lcm}{lcm} 
% blackboard bold
\newcommand{\RR}{\mathbb{R}}
\newcommand{\FF}{\mathbb{R}}
\newcommand{\QQ}{\mathbb{Q}}
\newcommand{\ZZ}{\mathbb{Z}}
\newcommand{\NN}{\mathbb{N}}
\newcommand{\CC}{\mathbb{C}}
% mathcal
\newcommand{\m}[1]{\mathcal{#1}}
% vectors
\newcommand{\vvec}[1]{\textbf{#1}} %
\newcommand{\ii}[0]{\vvec{i}} %
\newcommand{\jj}[0]{\vvec{j}} %
\newcommand{\kk}[0]{\vvec{k}} %
\newcommand{\hvec}[1]{\hat{\textbf{#1}}} %
\newcommand{\cvec}[3]{ %column vector
    \ensuremath{\left(\begin{array}{c}#1\\#2\\#3\end{array}\right)}}
\newcommand{\pfrac}[2]{\frac{\partial#1}{\partial#2}} %
\newcommand{\norm}[1]{\left\lVert#1\right\rVert} %
% dotp roduct
\makeatletter
\newcommand*\dotp{\mathpalette\dotp@{.5}}
\newcommand*\dotp@[2]{\mathbin{\vcenter{\hbox{\scalebox{#2}{$\m@th#1\bullet$}}}}}
\makeatother
% divrg and curl
\newcommand{\divrg}[0]{\nabla\dotp} %
\newcommand{\curl}[0]{\nabla\times} %

\title{Math 421 Notes}
\author{Henry Xia}
%\date{15 September 2017}

\begin{document}

\maketitle

\tableofcontents

%%%%%%%%%%%%%%%%%%%%%%%%%%%%%%%%%%%%%%%%%%%%%%%%%%%%%%%%%%%%%%%%%%%%%%
%%%%%%%%%%%%%%%%%%%%%%%%%%%%%%%%%%%%%%%%%%%%%%%%%%%%%%%%%%%%%%%%%%%%%%
% 2020 01 08

\section{Topological Spaces}

\begin{defn} A topological space $(S,\m{T})$ is a nonempty set with a family of subsets
$\m{T}$ such that
\begin{enumerate}
  \item $\emptyset\in\m{T}$
  \item $S\in\m{T}$
  \item $\m{T}$ is closed under finite intersections and arbitrary unions
\end{enumerate}
\end{defn}

Examples: $\set{\emptyset,S}$ (indiscrete topology), $2^S$ (discrete topology).

A metric on a metric space defines a topology. Not all topologies have a corresponding
metric. A topology is called metrizable if we can define a metric such that ``open'' has
the same meaning.

Topologies can be partially ordered. $\m{T}_1\prec\m{T}_2$ if $\m{T}_1\subset\m{T}_2$ as
sets. Denote $\m{T}(\m{E})$ to be the topology generated by $\m{E}\subset 2^S$.

\begin{defn}
A base of $\m{T}$ is a family $\m{B}\subset\m{T}$ such that for any nonempty open set
$O\in\m{T}$, there exists a colletion $\set{B_\alpha:B_\alpha\in\m{B}}$ such that
$O=\bigcup\alpha B_\alpha$.
\end{defn}

\begin{defn}
Let $(S,\m{T})$ be a topological space and $X\subset S$. Then, $\m{T}_x=\set{O\cap X:
O\subset\m{T}}$ is the relative topology $(X,\m{T}_x)$.
\end{defn}

\begin{defn}
A set $X$ is closed if $\exists Y\in\m{T}$ such that $X=Y^c$.
\end{defn}

\begin{defn}
The interior of $X$ is the largest open set $X^o\subset X$.
\end{defn}

\begin{defn}
The closure of $X$ is the smallest closed set $\overline{X}\supset X$.
\end{defn}

\begin{defn}
The boundary of $X$ is $\overline{X}\sm X^o$.
\end{defn}

\begin{defn}
A neighbourhood of $x\in S$ is a set $N_x\subset S$ such that $x\in N^o_x$
\end{defn}

\begin{defn}
A neighbourhood base of $x$ is a family $\m{N}_x$ such that each $N\in \m{N}_x$ is a
neighbourhood of $x$ and for any neighbourhood $M_x$, there exists some $N\in \m{N}_x$ such
that $N\subset M_x$.
\end{defn}

\begin{defn}[Classification of topological spaces]
A topological space is called $T_2$ or Hausdorff if $\forall x,y\in S$, $x\neq y$, there
exists $O_x,O_y\in\m{T}$ such that $x\in O_x$, $y\in O_y$, and $O_x\cap O_y = \emptyset$.
\end{defn}

\begin{defn}
A topological space $(S,\m{T})$ is
\begin{itemize}
\item separable if there exists a countable dense set
\item first countable if $\forall x\in S$, there exists a countable neighbourhood base
\item second countable if there exists a countable base
\end{itemize}
\end{defn}

\begin{prop}
Second countable implies both first countable and separable.
\end{prop}
\proof
(Second countable implies first countable)
Let $x\in S$, and let $M_x\subset\m{T}$ be a neighbourhood of $x$. Since $\m{B}$ is a base,
there exists open sets $N_\alpha\in\m{B}$ such that $\bigcup_\alpha N_\alpha = M_x^o$.
Observe that there exists some $N_\alpha$ such that $x\in N_\alpha$, whence second
countable.

(Second countable implies separable)
For each $B\in\m{B}$, choose some $x_B\in B$, and let $D=\bigcup_B x_B$. Suppose that
$\overline{D}\neq S$, then $\overline{D}^c$ is open. Since $\m{B}$ is a base, there exists
some $B\in\m{B}$ such that $B\subset\overline{D}^c$. Contradiction.
\qedhere

\begin{defn}
A sequence $\set{x_n}_{x\in\NN}$ in $(S,\m{T})$ is convergent if $\exists x\in S$ such that
for any neighbourhood of $x$, there exists some $N\in\NN$ such that $x_n\in N_x$ for all
$n>N$.
\end{defn}


%%%%%%%%%%%%%%%%%%%%%%%%%%%%%%%%%%%%%%%%%%%%%%%%%%%%%%%%%%%%%%%%%%%%%%
%%%%%%%%%%%%%%%%%%%%%%%%%%%%%%%%%%%%%%%%%%%%%%%%%%%%%%%%%%%%%%%%%%%%%%
% 2020 01 10

\begin{prop}
Let $(S,\m{T})$ be a first countable topological space, and $X\subset S$. Then
$x\in\overline{X}$ if and only if $x$ is the limit point of a convergent sequence
$\set{x_n}_{n\in\NN}\subset X$.
\end{prop}
\proof
Let $\m{N}_x=\set{O_n:n\in\NN}$ be a countable neighbourhood base of $x$ such that
$O_n\subset O_{n-1}$ for all $n\in\NN$. If $x\in\overline{X}$, then $O_n\cap
X\neq\emptyset$ for all $n\in\NN$. Then we can pick $x_n\in O_n\cap X$, whence $x_n\to x$.
Converse is similar.
\qedhere

\begin{defn}
Let $(S_1,\m{T}_1), (S_2,\m{T}_2)$ be topological spaces. A function $f:S_1\to S_2$ is
continuous if $f^{-1}(O)\in\m{T}_1$ for any $O\in\m{T}_2$. Ie. the preimage of any open set
is open.
\end{defn}

\begin{defn}
Let $(S_1,\m{T}_1), (S_2,\m{T}_2)$ be topological spaces. A function $f:S_1\to S_2$ is open
if $f(O)\in\m{T}_2$ for any $O\in\m{T}_1$.
\end{defn}

\begin{defn}
A homeomorphism is an invertible function that is open and continuous.
\end{defn}

\begin{defn}
Let $S_1$ be a set and let $(S_2,\m{T}_2)$ be a topological space. Let $\m{F}$ be a family
of functions from $S_1$ to $S_2$. Then, the topology on $S_1$ generated by
$\set{f^{-1}(O):O\in\m{T}_2}$ is called the $\m{F}$-weak topology.
\end{defn}

\begin{remark}
By definition, all functions $f\in\m{F}$ are continuous with respect to the above topology
on $S_1$.
\end{remark}

\begin{example}
Let $S_1=C([a,b];\RR)$ be the set of continuous functions, and let $S_2=\RR$ with the usual
metric topology. Let $E_x:S_1\to S_2$ where $E_x(f)=f(x)$ be the evaluation functions, and
let $\m{F}=\set{E_x:x\in[a,b]}$. The $\m{F}$-weak topology on $C([a,b];\RR)$ is the
topology of pointwise convergence.
\end{example}

\begin{defn}
A topological space $(S,\m{T})$ is compact if any open cover has a finite subcover.
\end{defn}

%%%%%%%%%%%%%%%%%%%%%%%%%%%%%%%%%%%%%%%%%%%%%%%%%%%%%%%%%%%%%%%%%%%%%%
%%%%%%%%%%%%%%%%%%%%%%%%%%%%%%%%%%%%%%%%%%%%%%%%%%%%%%%%%%%%%%%%%%%%%%
% 2020 01 13

\begin{defn}
A subset $X\subset S$ is compact if it is compact in the relative topology.
\end{defn}

\begin{defn}
A subset $X\subset S$ is precompact if its closure is compact.
\end{defn}

\begin{defn}
We say that $(S,\m{T})$ has the finite intersection property if for any family of closed
sets $\m{C}$ such that $\bigcap_{i=1}^nC_i\neq\emptyset$ for any finite subfamily
$\set{C_1,...,C_n}$ also satisfies $\bigcap_{C\in\m{C}}C\neq\emptyset$.
\end{defn}

\begin{exe}
$S$ is compact if and only if it has the finite intersection property.
\end{exe}

\begin{prop}
Let $X\subset S$ be a subset of a compact topological space $(S,\m{T})$. Then $X$ is
compact if $X$ is closed.
\end{prop}
\proof
Let $\m{C}$ be an open cover of $X$. Let $\m{C}'=\m{C}\cup\set{X^c}$ be an open cover of
$S$. There exists a finite subcover of $\m{C}'$, so there exists a finite subcover of $X$
(we can safely remove $X^c$ from the finite subcover of $S$ as $X\cap X^c=\emptyset$).
\qedhere

\begin{prop}
Let $(S_1,\m{T}_1)$ and $(S_2,\m{T}_2)$, and let $f:S_1\to S_2$ be continuous. If $S_1$ is
compact, then $f(S_1)\subset S_2$ is compact.
\end{prop}
\proof
Let $\m{C}$ be an open cover of $f(S_1)$. Let $\m{C}'=\set{f^{-1}(C):C\in\m{C}}$ be an open
cover of $S_1$ (preimages of open sets are open by continuity). Hence, there exists a
finite subcover of $S_1$, from which we get a finite subcover of $f(S_1)$.
\qedhere

\begin{prop}[Bolzano-Weierstrass property]
A second countable topological space is compact if and only if every sequence has a
convergent subsequence.
\end{prop}
\proof
Suppose that $S$ is compact, and suppose, for contradiction, that $\set{z_n}_{n\in\NN}$
does not have a convergent subsequence. Since $S$ is first countable, this means that for
any $x\in S$, there exists some neighbourhood $O_x$ of $x$ and some $N_x\in\NN$ such that
$z_n\notin O_x$ for all $n>N_x$.  Let $\m{C}=\set{O_x^o:x\in S}$ be an open cover of $S$.
Since $S$ is compact, there exists some finite subcover
$\m{C}'=\set{O_{x_1}^o,...,O_{x_m}^o}$. Then, let $N=\max\set{n_{x_1},...,n_{x_m}}$, whence
$z_n\notin\bigcup_iO_{x_i}=S$ for all $n>N$, which is a contradiction.

Suppose that every sequence of $S$ has a convergent subsequence. Since $S$ is second
countable, there exists a countable open cover $\m{C}=\set{O_i:i\in\NN}$. Suppose, for
contradiction, that $\m{C}$ has no finite subcover. Then, for any $i\in\NN$, there exists
some $x_i\notin\bigcup_{j=1}^iO_j$. Let $\set{x_{n_i}}_{i\in\NN}$ be a convergent
subsequence and let $x$ be its limit. Since $\m{C}$ is a cover, there exists $j$ such that
$x\in O_j$. It follows that there exists some $N\in\NN$ such that $x_{n_k}\in O_j$ for all
$k>N$, which is a contradiction.
\qedhere


%%%%%%%%%%%%%%%%%%%%%%%%%%%%%%%%%%%%%%%%%%%%%%%%%%%%%%%%%%%%%%%%%%%%%%
%%%%%%%%%%%%%%%%%%%%%%%%%%%%%%%%%%%%%%%%%%%%%%%%%%%%%%%%%%%%%%%%%%%%%%
% 2020 01 15

% SNOW


%%%%%%%%%%%%%%%%%%%%%%%%%%%%%%%%%%%%%%%%%%%%%%%%%%%%%%%%%%%%%%%%%%%%%%
%%%%%%%%%%%%%%%%%%%%%%%%%%%%%%%%%%%%%%%%%%%%%%%%%%%%%%%%%%%%%%%%%%%%%%
% 2020 01 17

\subsection{Weierstrass Theorems}

\begin{thm}[``Classical'' Weierstrass]
If $f$ is a continuous function on $[a,b]$, then there exists a sequence of polynomials
$\set{P_n}_{n\in\NN}$ such that $\lim_{n\to\infty}P_n=f$ uniformly.
\end{thm}

\begin{remark}
This theorem implies that the set of polynomials is dense in $C_\RR([a,b])$ (real-valued
continuous functions).
\end{remark}

\begin{defn}
Let $X$ be compact and Hausdorff, and let $C_\RR(X)$ be the set of real-valued continuous
functions on $X$ equipped with pointwise multiplication: $(fg)(x)=f(x)g(x)$. This is an
algebra.
\end{defn}

\begin{defn}
An algebra $\m{A}\subset C_\RR(X)$ separates points if for any $x\neq y$ in $X$, there
exists an $f\in\m{A}$ such that $f(x)\neq f(y)$.
\end{defn}

\begin{thm}[Stone-Weierstrass]
Let $X$ be a compact Hausdorff space. Let $\m{A}$ be a closed subalgebra (wrt
$\norm{\cdot}_\infty$) of $C_\RR(X)$ that seaprates points. Then, either $\m{A}=C_\RR(X)$
or $\exists x_0\in X$ such that $\m{A}=\set{f\in C_\RR(X):f(x_0)=0}$.
\end{thm}

\begin{remark}
If $\m{A}$ separates points and $1\in\m{A}$, we must have $\overline{\m{A}}=C_\RR(X)$.
Hence, any unital subalgebra $\m{A}$ of $C_\RR(X)$ is dense.
\end{remark}

\begin{defn}
Let $f$ and $g$ be functions on the same domain. Write $f\wedge g=\min\set{f,g}$ and $f\vee
g=\max\set{f,g}$.
\end{defn}

\begin{defn}
A family $\m{F}\subset C_\RR(X)$ is a lattice if any functions $f,g\in\m{F}$, we have
$f\wedge g\in\m{F}$ and $f\vee g\in\m{F}$.
\end{defn}

\begin{lemma}
Any closed unital subalgebra $\m{A}\subset C_\RR(X)$ is a lattice.
\end{lemma}
\proof
Observe that
\begin{align*}
  f\vee g &= \frac12\abs{f-g} + \frac12(f+g) \\
  f\wedge g &= -((-f)\vee(-g)) ,
\end{align*}
so it suffices to show that $f\in\m{A}$ means $\abs{f}\in\m{A}$. By the classical
Weierstrass theorem, there is a sequence of polynomials $\set{P_n}_{n\in\NN}$ such that
$\abs{P_n(x)-\abs{x}}<\frac1n$ for all $x\in[-1,1]$. Hence,
\[
\norm{P_n(h)-\abs{h}}_\infty<\frac1n \text{ where } h=\frac{f}{\norm{f}_\infty} ,
\]
that is $P_n(h)\to\abs{h}$ uniformly. Note that $P_n(h)\in\m{A}$, so we are done.
\qedhere

\begin{prop}[Kakutani-Klein]
If $\m{L}$ is a closed lattice that separates points such that $1\in\m{L}$. Then,
$\m{L}=C_\RR(X)$.
\end{prop}
\proof
Let $g\in C_\RR(X)$, let $\e>0$, and let $x\neq y\in X$. The map $\phi:\m{L}\to\RR^2$ such
that $h\mapsto(h(x),h(y))$ is an algebra homomorphism. The image contains $(1,1)$ since
$1\in\m{L}$. The image also contains $(a,b)$ with $a\neq b$ since $\m{L}$ separates points.
It suffices to look at the subalgebras of $\RR^2$, whence the image is all of $\RR^2$. Now,
there exists $f_{xy}\in\m{L}$ such that $f_{xy}(x)=g(x)$ and $f_{xy}(y)=g(y)$.

By continuity, there exists an open neighbourhood $N_y$ of $y$ such that
$f_{xy}(z)+\e=g(z)$ for all $z\in N_y$. By compactness, there exists a finite subcover
$\set{N_{y_1},...,N_{y_m}}\subset\set{N_y:y\in X}$. Let
$f_x=\max\set{f_{xy_1},...,f_xy_m}\in\m{L}$. Note that $f_x(x)=g(x)$ and $f_x(z)>g(z)-\e$
for all $z\in X$. This gives a lower bound. We can also find an upper bound whence there
exists some $f\in\m{L}$ such that $\norm{f-g}_\infty<2\e$.
\qedhere



%%%%%%%%%%%%%%%%%%%%%%%%%%%%%%%%%%%%%%%%%%%%%%%%%%%%%%%%%%%%%%%%%%%%%%
%%%%%%%%%%%%%%%%%%%%%%%%%%%%%%%%%%%%%%%%%%%%%%%%%%%%%%%%%%%%%%%%%%%%%%
% 2020 01 20

The Stone-Weierstrass Theorem has two generalizations.
\begin{enumerate}
\item It extends to complex-valued functions, provided the subalgebras are also closed
under conjugation. Any $f\in C_\CC(X)$ can be written as $f=(f+\overline{f})/2 -
i(f-\overline{f})/2$.
\item It extends to locally compact Hausdorff (LCH) spaces, that is topological spaces $S$
such that every $x\in S$ has a compact neighbourhood. Here, the relevant algebra is the set
of functions that vanish at infinity, that is $f\in C_\RR(S)$ such that
$\forall\epsilon>0$, the set $\set{x\in S:\abs{f(x)}\ge\epsilon}$ is compact. Hence,
consider $S\cup\set\infty$.
\end{enumerate}

\subsection{Hausdorff Spaces}

\begin{defn}
A space $S$ is locally compact if each $x\in S$ has a compact neighbourhood.
\end{defn}

\begin{lemma}
Let $(S,\m{T})$ be Hausdorff. Let $\set{x_n}_{n\in\NN}$ be a convergent sequence in $S$.
Then the limit $x=\lim_{n\to\infty}x_n$ is unique.
\end{lemma}
\proof
Let $y\neq x$. By Hausdorff, there exists disjoint open sets $O_x$ and $O_y$ such that
$x\in O_x$ and $y\in O_y$. Since $x_n\to x$, there exists $n_0\in\NN$ such that $x_n\in
O_x$ for all $n>n_0$. Hence $x_n\notin O_y$, so $\set{x_n}_{n\in\NN}$ does not converge to
$y$.
\qedhere

\begin{lemma}
Let $(S,\m{T})$ be Hausdorff, and let $K\subset S$ be compact. For any $x\notin K$, there
are open disjoint sets $U,V$ such that $x\in U$ and $K\subset V$.
\end{lemma}

\begin{prop}
Let $(S,\m{T})$ be Hausdorff, and let $K\subset S$ be compact. Then $K$ is closed.
\end{prop}

\begin{thm}
Let $(S_1,\m{T}_1)$ and $(S_2,\m{T}_2)$ be two compact Hausdorff spaces, and let $f:S_1\to
S_2$. If $f$ is continuous and bijective, then $f$ is a homeomorphism.
\end{thm}


%%%%%%%%%%%%%%%%%%%%%%%%%%%%%%%%%%%%%%%%%%%%%%%%%%%%%%%%%%%%%%%%%%%%%%
%%%%%%%%%%%%%%%%%%%%%%%%%%%%%%%%%%%%%%%%%%%%%%%%%%%%%%%%%%%%%%%%%%%%%%
% 2020 01 22

\begin{prop}
Let $S$ be a LCH space. Let $K\subset U\subset S$, where $K$ is compact and $U$ is open.
Then, there is an open set $O$ with compact closure such that
\[
K\subset O\subset \overline{O}\subset U .
\]
\end{prop}
\proof
Since $S$ is LCH, every point of $K$ has an open neighbourhood with compact closure. Since
$K$ is compact, there is a finite subcover of such neighbourhoods. Hence $K$ is a subset of
their union $V$ which has a compact closure. If $U=S$, then $O=V$. Otherwise, $U^c$
nonempty. By Hausdorff, for any $x\in U^c\subset K^c$, there exists an open set $O_x$ such
that $K\subset O_x$ and $x\notin \overline{O_x}$. It follows that
\[
\bigcap_{x\in U^c} U^c\cap\overline{V}\overline{O_x} = \emptyset .
\]
By the finite intersection property, there are finitely many $\set{x_1,...,x_n}$ such that
\[
U^c\cap\overline{V}\cap\overline{O_{x_1}}\cap\cdots\cap\overline{O_{x_n}} = \emptyset .
\]
Let $O=V\cap O_{x_1}\cap\cdots\cap O_{x_n}$.
\qedhere

\begin{defn}
The support of a complex-valued function $f$ is $\overline{\set{x\in S:f(x)\neq0}}$.
\end{defn}

\begin{defn}
Let $C_c(S)$ be the set of compactly supported continuous functions on $S$. Write $K\prec
f$ for a compact set $K$ and a function $f\in C_c(S)$ such that $0\le f(x)\le 1$ for all
$x\in S$ and $f(x)=1$ for all $x\in K$. Write $f\prec U$ for an open set $U$ and a function
$f\in C_c(S)$ such that $0\le f(x)\le 1$ for all $x\in S$ and the support of $f$ is a
subset of $U$.
\end{defn}

\begin{lemma}[Urysohn's Lemma]
Let $S$ be a LCH space. Let $K\subset U\subset S$, where $K$ is compact and $U$ is open.
There exists a function $f\in C_c(S)$ such that $K\prec f\prec U$.
\end{lemma}
\proof
We have a family of open sets $\set{O_r:r\in\QQ\cap[0,1]}$ with compact closures such that
$K\subset O_1$, $\overline{O_0}\subset U$, and $\overline{O_s}\subset O_r$ for $s>r$.
Define
\[
f_r(x) = r\chi_{O_r} = \begin{cases}
r&\text{if }x\in O_r \\
0&\text{otherwise}
\end{cases}
~~~\text{and}~~~
g_s(x) = s+(1-s)\chi_{\overline{O_s}} = \begin{cases}
1&\text{if }x\in\overline{O_s} \\
s&\text{otherwise}
\end{cases} .
\]
Also define
\[
f(x)=\sup\set{f_r(x):r\in\QQ\cap[0,1]}
~~~\text{and}~~~
g(x)=\inf\set{g_s(x):s\in\QQ\cap[0,1]} .
\]
Observe that $\set{x:f(x)>a}$ is open for all $a\in\RR$ (lower semicontinuous), and
$\set{x:g(x)<a}$ is open for all $a\in\RR$ (upper semicontinuous). Moreover, $0\le f\le1$
and $f(x)=1$ for all $x\in K\subset O_1$. Also, $\text{supp}(f)\subset \overline{O_0}
\subset U$. It suffices to show that $f=g$. Note that $f_r(x)>g_s(x)$ if $r>s$ and $x\in
O_r\cap\overline{O_s}^c$. But $r>s$ means $O_r\subset O_s$, so $f_r\le g_s$ for all
$r,s\in\QQ$, so $f\le g$. Suppose, for contradiction, that there exists an $x$ such that
$f(x)<g(x)$. Then, $\exists r,s\in\QQ$ such that $f(x)<r<s<g(x)$. The first inequality
implies that $x\notin O_r$, while the last implies that $x\in\overline{O_s}$, which is a
contradiction since $r<s$. Therefore $f=g$, and we are done.
\qedhere


%%%%%%%%%%%%%%%%%%%%%%%%%%%%%%%%%%%%%%%%%%%%%%%%%%%%%%%%%%%%%%%%%%%%%%
%%%%%%%%%%%%%%%%%%%%%%%%%%%%%%%%%%%%%%%%%%%%%%%%%%%%%%%%%%%%%%%%%%%%%%
% 2020 01 24

\begin{prop}
Let $(S,\m{T})$ be LCH, let $K$ be compact, and let $\set{O_i:1\le i\le n}$ be a finite
open cover of $K$. Then there are functions $\set{f_i\in C_c(S):1\le i\le n}$ such that
\begin{enumerate}
\item $f_i\prec O_i$ for $i\in[n]$.
\item $\sum_{i=1}^nf_i(x)=1$ for all $x\in K$.
\end{enumerate}
The set $\set{f_i}$ is called a partition of unity subordinate to the cover $\set{O_i}$.
\end{prop}
\proof
Let $x\in K$, there exists $i$ such that $x\in O_i$. There is a compact neighbourhood $N_x$
of $x$ such that $N_x\subset O_i$. By compactness, there exists $\set{x_1,...,x_m}$ such
that $K\subset\bigcup_{j=1}^mN_x^o\subset\bigcup_{j=1}^mN_x$. For $i\in[n]$, define
$K_i=\bigcup_{N_{x_j}\subset O_i}N_{x_j}$. By Urysohn's lemma, there exists continuous
functions $g_i$ such that $K_i\prec g_i\prec O_i$. Since $K\subset\bigcup_{i=1}^nK_i$, we
have $\sum_{i=1}^ng_i\ge1$ on $K$.

Let $W=\set{x:\sum_{i=1}^ng_i(x)>0}$ is open because $W$ is the preimage of an open set, so
by Urysohn's lemma, there exists $f$ such that $K\prec f\prec W$. Let $g_{n+1}=1-f$. By
construction, $G(x)=\sum_{i=1}^{n+1}g_i(x)>0$ everywhere. Now, let $f_i(x)=g_i(x)/G(x)$ for
$i\in[n]$. Therefore, $f_i\prec O_i$ (ie. $\text{supp}(f_i)\subset O_i$), and
$\sum_{i=1}^nf_i=1$ on $K$.
\qedhere

\begin{prop}[Tietze's Extension]
Let $(S,\m{T})$ be LCH, let $K$ be compact, and let $f\in C(K)$. Then there exists $F\in
C_c(S)$ such that $F(x)=f(x)$ for all $x\in K$.
\end{prop}
\proof
Since $f$ is continuous on a compact set, it is bounded. We assume WLOG that $\abs{f}\le1$
on $K$. Let $V$ be an open set with compact closure such that $K\subset V$. Let
$K^+=\set{x\in K:f(x)\ge\frac13}$, $K^-=\set{x\in K:f(x)\le-\frac13}$. Note that $K^+$ and
$K^-$ are disjoint closed (compact) subsets of $K$. Applying Urysohn's Lemma to $K^+$ and
$V\sm K^-$, and to $K^-$ and $V\sm K^+$, and rescaling gives $f_1\in C_c(S)$ such that
$f_1=\frac13$ on $K^+$, $f_1=-\frac13$ on $K^-$, and $\abs{f_1}\le\frac13$ everywhere.
Moreover, $\text{supp}(f)\subset V$ and $\abs{f-f_1}\le\frac23$ on $K$. Repeat this
procedure for $f-f_1$ to get $f_2$ such that $\abs{f-f_1-f_2}\le\left(\frac23\right)^2$ on
$K$. Hence, we obtain a sequence $\set{f_n}_{n\in\NN}$ such that $f_n\in C_c(S)$ where
$\abs{f_n}\le\frac13\left(\frac23\right)^{n-1}$ and
$\abs{f-\sum_{i=1}^n}\le\left(\frac23\right)^n$ on $K$. Now, let $F=\sum_{i=1}^\infty f_i$.
Note that $F$ is convergent uniformly everywhere so $F$ is continuous, and $F=f$ on $K$.
Moreover, $\text{supp}(F)\subset\overline{V}$.
\qedhere



%%%%%%%%%%%%%%%%%%%%%%%%%%%%%%%%%%%%%%%%%%%%%%%%%%%%%%%%%%%%%%%%%%%%%%
%%%%%%%%%%%%%%%%%%%%%%%%%%%%%%%%%%%%%%%%%%%%%%%%%%%%%%%%%%%%%%%%%%%%%%
% 2020 01 27




%%%%%%%%%%%%%%%%%%%%%%%%%%%%%%%%%%%%%%%%%%%%%%%%%%%%%%%%%%%%%%%%%%%%%%
%%%%%%%%%%%%%%%%%%%%%%%%%%%%%%%%%%%%%%%%%%%%%%%%%%%%%%%%%%%%%%%%%%%%%%
% 2020 01 29





%%%%%%%%%%%%%%%%%%%%%%%%%%%%%%%%%%%%%%%%%%%%%%%%%%%%%%%%%%%%%%%%%%%%%%
%%%%%%%%%%%%%%%%%%%%%%%%%%%%%%%%%%%%%%%%%%%%%%%%%%%%%%%%%%%%%%%%%%%%%%
% 2020 01 31





\end{document}
























