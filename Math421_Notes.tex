\documentclass[11pt]{article}
\usepackage{hyperref}
\usepackage[margin=1in]{geometry}
\usepackage{amsmath}
\usepackage{amsthm}
\usepackage{amssymb}
\usepackage{amsfonts}
\usepackage{graphicx} %\usepackage[pdftex]{graphicx}
\usepackage{xcolor}
\usepackage{setspace}
%\usepackage{tikz}


\setlength\parindent{0pt}

\newtheorem{thm}{Theorem}[section]
\newtheorem{cor}[thm]{Corollary}
\newtheorem{lemma}[thm]{Lemma}
\theoremstyle{definition}
\newtheorem{defn}[thm]{Definition}
\newtheorem{example}[thm]{Example}
\newtheorem{exe}[thm]{Exercise}
\newtheorem{prop}[thm]{Proposition}
\newtheorem{pty}[thm]{Property}
\newtheorem{remark}[thm]{Remark}
\newtheorem{obs}[thm]{Observation}
\newcommand{\The}[2]{\begin{#1}#2\end{#1}}

\newcommand{\ord}[0]{\text{ord}}

% notes
\iftrue
\newcommand{\f}[2]{\frac{#1}{#2}}
\newcommand{\re}[1]{\frac{1}{#1}}
\newcommand{\half}[0]{\frac{1}{2}}
\newcommand{\ift}[0]{It follows that}
\newcommand{\cp}[1]{\overline{#1}}
\newcommand{\Note}[0]{\noindent\textbf{Note: }}
\newcommand{\Claim}[0]{\noindent\textbf{Claim: }}
\newcommand{\Lemma}[1]{\noindent\textbf{Lemma #1}: } %
\newcommand{\Ex}[0]{\noindent\textbf{Example: }} %
\newcommand{\Special}[0]{\noindent\textbf{Special case: }} %
\newcommand{\solution}[2]{\item[]\proof[Solution to #1] #2 \qedhere}
\newcommand{\legendre}[2]{\left(\frac{#1}{#2}\right)}
\newcommand{\dent}[0]{\hspace{0.5in}}
\fi

\newcommand{\sm}[0]{\setminus}
\newcommand{\set}[1]{\left\{ #1 \right\}}
\newcommand{\nl}[0]{\vspace{12pt}}
\newcommand{\rng}[2]{#1,\dots,#2}
\newcommand{\srng}[3]{#1_#2,\dots,#1_#3}
\newcommand{\st}[0]{\text{ such that }}
\newcommand{\et}[0]{\text{ and }}
\newcommand{\then}[0]{\text{ then }}
\newcommand{\forsome}[0]{\text{ for some }}
\newcommand{\floor}[1]{\lfloor #1 \rfloor}

% misc
\newcommand{\abs}[1]{\left\lvert#1\right\rvert} %
% lcm ???
\DeclareMathOperator{\lcm}{lcm} 
% blackboard bold
\newcommand{\RR}{\mathbb{R}}
\newcommand{\FF}{\mathbb{R}}
\newcommand{\QQ}{\mathbb{Q}}
\newcommand{\ZZ}{\mathbb{Z}}
\newcommand{\NN}{\mathbb{N}}
\newcommand{\CC}{\mathbb{C}}
% mathcal
\newcommand{\m}[1]{\mathcal{#1}}
% vectors
\newcommand{\vvec}[1]{\textbf{#1}} %
\newcommand{\ii}[0]{\vvec{i}} %
\newcommand{\jj}[0]{\vvec{j}} %
\newcommand{\kk}[0]{\vvec{k}} %
\newcommand{\hvec}[1]{\hat{\textbf{#1}}} %
\newcommand{\cvec}[3]{ %column vector
    \ensuremath{\left(\begin{array}{c}#1\\#2\\#3\end{array}\right)}}
\newcommand{\pfrac}[2]{\frac{\partial#1}{\partial#2}} %
\newcommand{\norm}[1]{\left\lVert#1\right\rVert} %
% dotp roduct
\makeatletter
\newcommand*\dotp{\mathpalette\dotp@{.5}}
\newcommand*\dotp@[2]{\mathbin{\vcenter{\hbox{\scalebox{#2}{$\m@th#1\bullet$}}}}}
\makeatother
% divrg and curl
\newcommand{\divrg}[0]{\nabla\dotp} %
\newcommand{\curl}[0]{\nabla\times} %

\title{Math 421 Notes}
\author{Henry Xia}
%\date{15 September 2017}

\begin{document}

\maketitle

\tableofcontents

%%%%%%%%%%%%%%%%%%%%%%%%%%%%%%%%%%%%%%%%%%%%%%%%%%%%%%%%%%%%%%%%%%%%%%
%%%%%%%%%%%%%%%%%%%%%%%%%%%%%%%%%%%%%%%%%%%%%%%%%%%%%%%%%%%%%%%%%%%%%%
% 2020 01 08

\section{Introduction}

\begin{defn} A topological space $(S,\m{T})$ is a nonempty set with a family of subsets
$\m{T}$ such that
\begin{enumerate}
  \item $\emptyset\in\m{T}$
  \item $S\in\m{T}$
  \item $\m{T}$ is closed under finite intersections and arbitrary unions
\end{enumerate}
\end{defn}

Examples: $\set{\emptyset,S}$ (indiscrete topology), $2^S$ (discrete topology).

A metric on a metric space defines a topology. Not all topologies have a corresponding
metric. A topology is called metrizable if we can define a metric such that ``open'' has
the same meaning.

Topologies can be partially ordered. $\m{T}_1\prec\m{T}_2$ if $\m{T}_1\subset\m{T}_2$ as
sets. Denote $\m{T}(\m{E})$ to be the topology generated by $\m{E}\subset 2^S$.

\begin{defn}
A base of $\m{T}$ is a family $\m{B}\subset\m{T}$ such that for any nonempty open set
$O\in\m{T}$, there exists a colletion $\set{B_\alpha:B_\alpha\in\m{B}}$ such that
$O=\bigcup\alpha B_\alpha$.
\end{defn}

\begin{defn}
Let $(S,\m{T})$ be a topological space and $X\subset S$. Then, $\m{T}_x=\set{O\cap X:
O\subset\m{T}}$ is the relative topology $(X,\m{T}_x)$.
\end{defn}

\begin{defn}
A set $X$ is closed if $\exists Y\in\m{T}$ such that $X=Y^c$.
\end{defn}

\begin{defn}
The interior of $X$ is the largest open set $X^o\subset X$.
\end{defn}

\begin{defn}
The closure of $X$ is the smallest closed set $\overline{X}\supset X$.
\end{defn}

\begin{defn}
The boundary of $X$ is $\overline{X}\sm X^o$.
\end{defn}

\begin{defn}
A neighbourhood of $x\in S$ is a set $N_x\subset S$ such that $x\in N^o_x$
\end{defn}

\begin{defn}
A neighbourhood base of $x$ is a family $\m{N}_x$ such that each $N\in \m{N}_x$ is a
neighbourhood of $x$ and for any neighbourhood $M_x$, there exists some $N\in \m{N}_x$ such
that $N\subset M_x$.
\end{defn}

\begin{defn}[Classification of topological spaces]
A topological space is called $T_2$ or Hausdorff if $\forall x,y\in S$, $x\neq y$, there
exists $O_x,O_y\in\m{T}$ such that $x\in O_x$, $y\in O_y$, and $O_x\cap O_y = \emptyset$.
\end{defn}

\begin{defn}
A topological space $(S,\m{T})$ is
\begin{itemize}
\item separable if there exists a countable dense set
\item first countable if $\forall x\in S$, there exists a countable neighbourhood base
\item second countable if there exists a countable base
\end{itemize}
\end{defn}

\begin{prop}
Second countable implies both first countable and separable.
\end{prop}
\proof
(Second countable implies first countable)
Let $x\in S$, and let $M_x\subset\m{T}$ be a neighbourhood of $x$. Since $\m{B}$ is a base,
there exists open sets $N_\alpha\in\m{B}$ such that $\bigcup_\alpha N_\alpha = M_x^o$.
Observe that there exists some $N_\alpha$ such that $x\in N_\alpha$, whence second
countable.

(Second countable implies separable)
For each $B\in\m{B}$, choose some $x_B\in B$, and let $D=\bigcup_B x_B$. Suppose that
$\overline{D}\neq S$, then $\overline{D}^c$ is open. Since $\m{B}$ is a base, there exists
some $B\in\m{B}$ such that $B\subset\overline{D}^c$. Contradiction.
\qedhere

\begin{defn}
A sequence $\set{x_n}_{x\in\NN}$ in $(S,\m{T})$ is convergent if $\exists x\in S$ such that
for any neighbourhood of $x$, there exists some $N\in\NN$ such that $x_n\in N_x$ for all
$n>N$.
\end{defn}


%%%%%%%%%%%%%%%%%%%%%%%%%%%%%%%%%%%%%%%%%%%%%%%%%%%%%%%%%%%%%%%%%%%%%%
%%%%%%%%%%%%%%%%%%%%%%%%%%%%%%%%%%%%%%%%%%%%%%%%%%%%%%%%%%%%%%%%%%%%%%
% 2020 01 10

\begin{prop}
Let $(S,\m{T})$ be a first countable topological space, and $X\subset S$. Then
$x\in\overline{X}$ if and only if $x$ is the limit point of a convergent sequence
$\set{x_n}_{n\in\NN}\subset X$.
\end{prop}
\proof
Let $\m{N}_x=\set{O_n:n\in\NN}$ be a countable neighbourhood base of $x$ such that
$O_n\subset O_{n-1}$ for all $n\in\NN$. If $x\in\overline{X}$, then $O_n\cap
X\neq\emptyset$ for all $n\in\NN$. Then we can pick $x_n\in O_n\cap X$, whence $x_n\to x$.
Converse is similar.
\qedhere

\begin{defn}
Let $(S_1,\m{T}_1), (S_2,\m{T}_2)$ be topological spaces. A function $f:S_1\to S_2$ is
continuous if $f^{-1}(O)\in\m{T}_1$ for any $O\in\m{T}_2$. Ie. the preimage of any open set
is open.
\end{defn}

\begin{defn}
Let $(S_1,\m{T}_1), (S_2,\m{T}_2)$ be topological spaces. A function $f:S_1\to S_2$ is open
if $f(O)\in\m{T}_2$ for any $O\in\m{T}_1$.
\end{defn}

\begin{defn}
A homeomorphism is an invertible function that is open and continuous.
\end{defn}

\begin{defn}
Let $S_1$ be a set and let $(S_2,\m{T}_2)$ be a topological space. Let $\m{F}$ be a family
of functions from $S_1$ to $S_2$. Then, the topology on $S_1$ generated by
$\set{f^{-1}(O):O\in\m{T}_2}$ is called the $\m{F}$-weak topology.
\end{defn}

\begin{remark}
By definition, all functions $f\in\m{F}$ are continuous with respect to the above topology
on $S_1$.
\end{remark}

\begin{example}
Let $S_1=C([a,b];\RR)$ be the set of continuous functions, and let $S_2=\RR$ with the usual
metric topology. Let $E_x:S_1\to S_2$ where $E_x(f)=f(x)$ be the evaluation functions, and
let $\m{F}=\set{E_x:x\in[a,b]}$. The $\m{F}$-weak topology on $C([a,b];\RR)$ is the
topology of pointwise convergence.
\end{example}

\begin{defn}
A topological space $(S,\m{T})$ is compact if any open cover has a finite subcover.
\end{defn}

%%%%%%%%%%%%%%%%%%%%%%%%%%%%%%%%%%%%%%%%%%%%%%%%%%%%%%%%%%%%%%%%%%%%%%
%%%%%%%%%%%%%%%%%%%%%%%%%%%%%%%%%%%%%%%%%%%%%%%%%%%%%%%%%%%%%%%%%%%%%%
% 2020 01 13

\begin{defn}
A subset $X\subset S$ is compact if it is compact in the relative topology.
\end{defn}

\begin{defn}
A subset $X\subset S$ is precompact if its closure is compact.
\end{defn}

\begin{defn}
We say that $(S,\m{T})$ has the finite intersection property if for any family of closed
sets $\m{C}$ such that $\bigcap_{i=1}^nC_i\neq\emptyset$ for any finite subfamily
$\set{C_1,...,C_n}$ also satisfies $\bigcap_{C\in\m{C}}C\neq\emptyset$.
\end{defn}

\begin{exe}
$S$ is compact if and only if it has the finite intersection property.
\end{exe}

\begin{prop}
Let $X\subset S$ be a subset of a compact topological space $(S,\m{T})$. Then $X$ is
compact if $X$ is closed.
\end{prop}
\proof
Let $\m{C}$ be an open cover of $X$. Let $\m{C}'=\m{C}\cup\set{X^c}$ be an open cover of
$S$. There exists a finite subcover of $\m{C}'$, so there exists a finite subcover of $X$
(we can safely remove $X^c$ from the finite subcover of $S$ as $X\cap X^c=\emptyset$).
\qedhere

\begin{prop}
Let $(S_1,\m{T}_1)$ and $(S_2,\m{T}_2)$, and let $f:S_1\to S_2$ be continuous. If $S_1$ is
compact, then $f(S_1)\subset S_2$ is compact.
\end{prop}
\proof
Let $\m{C}$ be an open cover of $f(S_1)$. Let $\m{C}'=\set{f^{-1}(C):C\in\m{C}}$ be an open
cover of $S_1$ (preimages of open sets are open by continuity). Hence, there exists a
finite subcover of $S_1$, from which we get a finite subcover of $f(S_1)$.
\qedhere

\begin{prop}[Bolzano-Weierstrass property]
A second countable topological space is compact if and only if every sequence has a
convergent subsequence.
\end{prop}
\proof
Suppose that $S$ is compact, and suppose, for contradiction, that $\set{z_n}_{n\in\NN}$
does not have a convergent subsequence. Since $S$ is first countable, this means that for
any $x\in S$, there exists some neighbourhood $O_x$ of $x$ and some $N_x\in\NN$ such that
$z_n\notin O_x$ for all $n>N_x$.  Let $\m{C}=\set{O_x^o:x\in S}$ be an open cover of $S$.
Since $S$ is compact, there exists some finite subcover
$\m{C}'=\set{O_{x_1}^o,...,O_{x_m}^o}$. Then, let $N=\max\set{n_{x_1},...,n_{x_m}}$, whence
$z_n\notin\bigcup_iO_{x_i}=S$ for all $n>N$, which is a contradiction.

Suppose that every sequence of $S$ has a convergent subsequence. Since $S$ is second
countable, there exists a countable open cover $\m{C}=\set{O_i:i\in\NN}$. Suppose, for
contradiction, that $\m{C}$ has no finite subcover. Then, for any $i\in\NN$, there exists
some $x_i\notin\bigcup_{j=1}^iO_j$. Let $\set{x_{n_i}}_{i\in\NN}$ be a convergent
subsequence and let $x$ be its limit. Since $\m{C}$ is a cover, there exists $j$ such that
$x\in O_j$. It follows that there exists some $N\in\NN$ such that $x_{n_k}\in O_j$ for all
$k>N$, which is a contradiction.
\qedhere


%%%%%%%%%%%%%%%%%%%%%%%%%%%%%%%%%%%%%%%%%%%%%%%%%%%%%%%%%%%%%%%%%%%%%%
%%%%%%%%%%%%%%%%%%%%%%%%%%%%%%%%%%%%%%%%%%%%%%%%%%%%%%%%%%%%%%%%%%%%%%
% 2020 01 15





%%%%%%%%%%%%%%%%%%%%%%%%%%%%%%%%%%%%%%%%%%%%%%%%%%%%%%%%%%%%%%%%%%%%%%
%%%%%%%%%%%%%%%%%%%%%%%%%%%%%%%%%%%%%%%%%%%%%%%%%%%%%%%%%%%%%%%%%%%%%%
% 2020 01 17






\end{document}
























